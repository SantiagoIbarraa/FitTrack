\documentclass[12pt,a4paper]{article}
\usepackage[utf8]{inputenc}
\usepackage[spanish]{babel}
\usepackage{geometry}
\usepackage{graphicx}
\usepackage{hyperref}
\usepackage{listings}
\usepackage{xcolor}
\usepackage{booktabs}
\usepackage{array}
\usepackage{longtable}
\usepackage{fancyhdr}
\usepackage{titlesec}
\usepackage{enumitem}
\usepackage{amsmath}
\usepackage{amsfonts}
\usepackage{amssymb}
\usepackage{float}

% Configuración de página
\geometry{margin=2.5cm}
\pagestyle{fancy}
\fancyhf{}
\fancyhead[L]{Modulo de Gimnasio - FitTrack}
\fancyhead[R]{\thepage}
\renewcommand{\headrulewidth}{0.4pt}
\setlength{\headheight}{15pt}

% Configuración de colores para código
\definecolor{codegreen}{rgb}{0,0.6,0}
\definecolor{codegray}{rgb}{0.5,0.5,0.5}
\definecolor{codepurple}{rgb}{0.58,0,0.82}
\definecolor{backcolour}{rgb}{0.95,0.95,0.92}
\definecolor{bluekeywords}{rgb}{0.13,0.13,1}
\definecolor{greencomments}{rgb}{0,0.5,0}
\definecolor{redstrings}{rgb}{0.9,0,0}

% Configuración de listings
\lstdefinestyle{mystyle}{
    backgroundcolor=\color{backcolour},   
    commentstyle=\color{greencomments},
    keywordstyle=\color{bluekeywords},
    numberstyle=\tiny\color{codegray},
    stringstyle=\color{redstrings},
    basicstyle=\ttfamily\footnotesize,
    breakatwhitespace=false,         
    breaklines=true,                 
    captionpos=b,                    
    keepspaces=true,                 
    numbers=left,                    
    numbersep=5pt,                  
    showspaces=false,                
    showstringspaces=false,
    showtabs=false,                  
    tabsize=2,
    frame=single,
    rulecolor=\color{codegray}
}

\lstdefinestyle{sqlstyle}{
    backgroundcolor=\color{backcolour},   
    commentstyle=\color{greencomments},
    keywordstyle=\color{bluekeywords},
    numberstyle=\tiny\color{codegray},
    stringstyle=\color{redstrings},
    basicstyle=\ttfamily\footnotesize,
    breakatwhitespace=false,         
    breaklines=true,                 
    captionpos=b,                    
    keepspaces=true,                 
    numbers=left,                    
    numbersep=5pt,                  
    showspaces=false,                
    showstringspaces=false,
    showtabs=false,                  
    tabsize=2,
    frame=single,
    rulecolor=\color{codegray},
    language=SQL
}

\lstset{style=mystyle}

% Configuración de títulos
\titleformat{\section}{\Large\bfseries}{\thesection}{1em}{}
\titleformat{\subsection}{\large\bfseries}{\thesubsection}{1em}{}
\titleformat{\subsubsection}{\normalsize\bfseries}{\thesubsubsection}{1em}{}

% Configuración de hipervínculos
\hypersetup{
    colorlinks=true,
    linkcolor=blue,
    filecolor=magenta,      
    urlcolor=cyan,
    citecolor=red,
}

\begin{document}

% Portada
\begin{titlepage}
\centering
\vspace*{2cm}

{\Huge\bfseries Modulo de Gimnasio}\\[0.5cm]
{\LARGE FitTrack}\\[1cm]

{\large Documentacion Tecnica Completa}\\[2cm]

\begin{minipage}{0.8\textwidth}
\centering
Este documento proporciona una guia tecnica exhaustiva del modulo de gimnasio de FitTrack, incluyendo arquitectura, implementacion, base de datos, componentes React, Server Actions y ejemplos practicos de uso.
\end{minipage}

\vfill

{\large Version 1.0}\\[0.5cm]
{\large \today}

\end{titlepage}

\tableofcontents
\newpage

\section{Introduccion al Modulo de Gimnasio}

El modulo de gimnasio es uno de los componentes mas complejos y funcionales de FitTrack. Permite a los usuarios registrar entrenamientos individuales, crear y gestionar rutinas personalizadas, hacer seguimiento de su progreso y acceder a un catalogo de ejercicios administrado.

\subsection{Caracteristicas Principales}

\begin{itemize}
    \item \textbf{Registro de Entrenamientos}: Permite registrar ejercicios individuales con peso, repeticiones y series
    \item \textbf{Gestion de Rutinas}: Crear, editar y ejecutar rutinas personalizadas
    \item \textbf{Catalogo de Ejercicios}: Base de datos de ejercicios administrada por administradores
    \item \textbf{Historial de Progreso}: Seguimiento detallado del progreso a lo largo del tiempo
    \item \textbf{Metricas y Estadisticas}: Analisis de rendimiento y tendencias
    \item \textbf{Selector de Ejercicios}: Interfaz intuitiva para seleccionar ejercicios del catalogo
    \item \textbf{Imagenes de Ejercicios}: Soporte para imagenes en ejercicios personalizados
\end{itemize}

\subsection{Arquitectura General}

El modulo sigue una arquitectura de capas bien definida:

\begin{enumerate}
    \item \textbf{Capa de Presentacion}: Componentes React con TypeScript
    \item \textbf{Capa de Logica}: Server Actions de Next.js
    \item \textbf{Capa de Datos}: Supabase PostgreSQL con RLS
    \item \textbf{Capa de Servicios}: Utilidades y helpers
\end{enumerate}

\section{Estructura de Base de Datos}

\subsection{Diagrama de Relaciones}

\begin{figure}[H]
\centering
\begin{verbatim}
    auth.users
         |
    +----+----+
    |         |
gym_workouts routines
    |         |
    |    routine_exercises
    |
gym_exercises
\end{verbatim}
\caption{Diagrama de relaciones de las tablas del modulo de gimnasio}
\end{figure}

\subsection{Tabla gym\_workouts}

Esta tabla almacena los entrenamientos individuales realizados por los usuarios.

\begin{lstlisting}[style=sqlstyle, caption=Estructura completa de gym_workouts]
CREATE TABLE IF NOT EXISTS public.gym_workouts (
    id UUID PRIMARY KEY DEFAULT uuid_generate_v4(),
    user_id UUID REFERENCES auth.users(id) ON DELETE CASCADE NOT NULL,
    exercise_name TEXT NOT NULL,
    weight_kg NUMERIC(5,2),
    repetitions INTEGER,
    sets INTEGER,
    image_url TEXT,
    created_at TIMESTAMP WITH TIME ZONE DEFAULT NOW()
);

-- Indices para optimizacion
CREATE INDEX IF NOT EXISTS idx_gym_workouts_user_id ON public.gym_workouts(user_id);
CREATE INDEX IF NOT EXISTS idx_gym_workouts_created_at ON public.gym_workouts(created_at);
CREATE INDEX IF NOT EXISTS idx_gym_workouts_exercise_name ON public.gym_workouts(exercise_name);
\end{lstlisting}

\textbf{Descripcion de campos:}
\begin{itemize}
    \item \texttt{id}: Identificador unico UUID generado automaticamente
    \item \texttt{user\_id}: Referencia al usuario propietario del entrenamiento
    \item \texttt{exercise\_name}: Nombre del ejercicio realizado (texto libre)
    \item \texttt{weight\_kg}: Peso utilizado en kilogramos (opcional, maximo 999.99)
    \item \texttt{repetitions}: Numero de repeticiones realizadas (opcional)
    \item \texttt{sets}: Numero de series realizadas (opcional)
    \item \texttt{image\_url}: URL de imagen del ejercicio (opcional)
    \item \texttt{created\_at}: Timestamp de creacion automatico
\end{itemize}

\subsection{Tabla routines}

Almacena las rutinas personalizadas creadas por los usuarios.

\begin{lstlisting}[style=sqlstyle, caption=Estructura completa de routines]
CREATE TABLE IF NOT EXISTS public.routines (
    id UUID DEFAULT gen_random_uuid() PRIMARY KEY,
    user_id UUID REFERENCES auth.users(id) ON DELETE CASCADE NOT NULL,
    name TEXT NOT NULL,
    description TEXT,
    created_at TIMESTAMP WITH TIME ZONE DEFAULT TIMEZONE('utc'::text, NOW()) NOT NULL,
    updated_at TIMESTAMP WITH TIME ZONE DEFAULT TIMEZONE('utc'::text, NOW()) NOT NULL
);

-- Indices para optimizacion
CREATE INDEX IF NOT EXISTS idx_routines_user_id ON public.routines(user_id);
CREATE INDEX IF NOT EXISTS idx_routines_created_at ON public.routines(created_at);
\end{lstlisting}

\textbf{Descripcion de campos:}
\begin{itemize}
    \item \texttt{id}: Identificador unico UUID de la rutina
    \item \texttt{user\_id}: Referencia al usuario propietario
    \item \texttt{name}: Nombre de la rutina (requerido)
    \item \texttt{description}: Descripcion opcional de la rutina
    \item \texttt{created\_at}: Timestamp de creacion
    \item \texttt{updated\_at}: Timestamp de ultima modificacion
\end{itemize}

\subsection{Tabla routine\_exercises}

Contiene los ejercicios que pertenecen a cada rutina con su orden especifico.

\begin{lstlisting}[style=sqlstyle, caption=Estructura completa de routine_exercises]
CREATE TABLE IF NOT EXISTS public.routine_exercises (
    id UUID DEFAULT gen_random_uuid() PRIMARY KEY,
    routine_id UUID REFERENCES public.routines(id) ON DELETE CASCADE NOT NULL,
    exercise_name TEXT NOT NULL,
    weight DECIMAL(6,2) NOT NULL CHECK (weight >= 0),
    repetitions INTEGER NOT NULL CHECK (repetitions > 0),
    sets INTEGER NOT NULL CHECK (sets > 0),
    image_url TEXT,
    order_index INTEGER NOT NULL DEFAULT 0,
    created_at TIMESTAMP WITH TIME ZONE DEFAULT TIMEZONE('utc'::text, NOW()) NOT NULL
);

-- Indices para optimizacion
CREATE INDEX IF NOT EXISTS idx_routine_exercises_routine_id ON public.routine_exercises(routine_id);
CREATE INDEX IF NOT EXISTS idx_routine_exercises_order ON public.routine_exercises(routine_id, order_index);
\end{lstlisting}

\textbf{Descripcion de campos:}
\begin{itemize}
    \item \texttt{id}: Identificador unico del ejercicio en la rutina
    \item \texttt{routine\_id}: Referencia a la rutina padre
    \item \texttt{exercise\_name}: Nombre del ejercicio
    \item \texttt{weight}: Peso en kilogramos (requerido, >= 0)
    \item \texttt{repetitions}: Repeticiones (requerido, > 0)
    \item \texttt{sets}: Series (requerido, > 0)
    \item \texttt{image\_url}: URL de imagen opcional
    \item \texttt{order\_index}: Orden del ejercicio en la rutina
    \item \texttt{created\_at}: Timestamp de creacion
\end{itemize}

\subsection{Tabla gym\_exercises}

Catalogo de ejercicios administrado por administradores del sistema.

\begin{lstlisting}[style=sqlstyle, caption=Estructura completa de gym_exercises]
CREATE TABLE IF NOT EXISTS gym_exercises (
    id UUID PRIMARY KEY DEFAULT gen_random_uuid(),
    name TEXT NOT NULL,
    category TEXT NOT NULL CHECK (category IN (
        'Pecho', 'Biceps', 'Triceps', 'Hombros', 
        'Pierna', 'Espalda', 'Otros'
    )),
    description TEXT,
    image_url TEXT,
    created_at TIMESTAMP WITH TIME ZONE DEFAULT NOW(),
    updated_at TIMESTAMP WITH TIME ZONE DEFAULT NOW()
);

-- Indices para optimizacion
CREATE INDEX IF NOT EXISTS idx_gym_exercises_category ON gym_exercises(category);
CREATE INDEX IF NOT EXISTS idx_gym_exercises_name ON gym_exercises(name);
\end{lstlisting}

\textbf{Descripcion de campos:}
\begin{itemize}
    \item \texttt{id}: Identificador unico del ejercicio
    \item \texttt{name}: Nombre del ejercicio
    \item \texttt{category}: Categoria del ejercicio (constraint de valores validos)
    \item \texttt{description}: Descripcion detallada del ejercicio
    \item \texttt{image\_url}: URL de imagen del ejercicio
    \item \texttt{created\_at}: Timestamp de creacion
    \item \texttt{updated\_at}: Timestamp de ultima modificacion
\end{itemize}

\subsection{Politicas de Seguridad (RLS)}

Todas las tablas implementan Row Level Security para garantizar que los usuarios solo accedan a sus propios datos.

\begin{lstlisting}[style=sqlstyle, caption=Politicas RLS para gym_workouts]
-- Habilitar RLS
ALTER TABLE public.gym_workouts ENABLE ROW LEVEL SECURITY;

-- Politicas para gym_workouts
CREATE POLICY "Usuarios pueden ver sus propios entrenamientos"
ON public.gym_workouts
FOR SELECT USING (auth.uid() = user_id);

CREATE POLICY "Usuarios pueden insertar sus propios entrenamientos"
ON public.gym_workouts
FOR INSERT WITH CHECK (auth.uid() = user_id);

CREATE POLICY "Usuarios pueden modificar sus propios entrenamientos"
ON public.gym_workouts
FOR UPDATE USING (auth.uid() = user_id);

CREATE POLICY "Usuarios pueden eliminar sus propios entrenamientos"
ON public.gym_workouts
FOR DELETE USING (auth.uid() = user_id);
\end{lstlisting}

\section{Server Actions - Logica de Negocio}

Las Server Actions manejan toda la logica de negocio del modulo de gimnasio. Estan implementadas en TypeScript con validacion robusta y manejo de errores.

\subsection{Archivo gym-actions.ts}

\subsubsection{createWorkout - Crear Entrenamiento}

\begin{lstlisting}[caption=Funcion createWorkout completa]
"use server"

import { revalidatePath } from "next/cache"
import { createClient } from "@/lib/supabase/server"

export async function createWorkout(prevState: any, formData: FormData) {
  // Extraer datos del formulario
  const exercise_name = formData.get("exercise_name")?.toString()
  const weight_kg = formData.get("weight_kg")?.toString()
  const repetitions = formData.get("repetitions")?.toString()
  const sets = formData.get("sets")?.toString()
  const image_url = formData.get("image_url")?.toString()

  // Validacion basica
  if (!exercise_name) {
    return { error: "El nombre del ejercicio es requerido" }
  }

  // Verificar autenticacion
  const supabase = await createClient()
  const {
    data: { user },
  } = await supabase.auth.getUser()

  if (!user) {
    return { error: "Usuario no autenticado" }
  }

  try {
    // Preparar datos para insercion
    const insertData = {
      user_id: user.id,
      exercise_name,
      weight_kg: weight_kg && weight_kg.trim() !== "" ? 
        Math.max(0, Number.parseFloat(weight_kg)) : null,
      repetitions: repetitions && repetitions.trim() !== "" ? 
        Math.max(1, Number.parseInt(repetitions)) : null,
      sets: sets && sets.trim() !== "" ? 
        Math.max(1, Number.parseInt(sets)) : null,
      image_url: image_url && image_url.trim() !== "" ? 
        image_url.trim() : null,
    }

    // Insertar en base de datos
    const { error } = await supabase
      .from("gym_workouts")
      .insert(insertData)

    if (error) {
      console.error("Database error:", error)
      return { error: "Error al guardar el ejercicio" }
    }

    // Revalidar cache
    revalidatePath("/gym")
    return { success: true }
  } catch (error) {
    console.error("Error:", error)
    return { error: "Error al guardar el ejercicio" }
  }
}
\end{lstlisting}

\textbf{Caracteristicas importantes:}
\begin{itemize}
    \item \textbf{Validacion de entrada}: Verifica que el nombre del ejercicio este presente
    \item \textbf{Autenticacion}: Confirma que el usuario este autenticado
    \item \textbf{Sanitizacion}: Convierte y valida valores numericos
    \item \textbf{Manejo de errores}: Captura y reporta errores de base de datos
    \item \textbf{Revalidacion}: Actualiza el cache de Next.js
\end{itemize}

\section{Componentes React}

\subsection{Pagina Principal - GymPage}

El componente principal que coordina toda la funcionalidad del modulo de gimnasio.

\begin{lstlisting}[caption=app/gym/page.tsx - Estructura completa]
"use client"

import { useState } from "react"
import { Dumbbell, ArrowLeft } from "lucide-react"
import Link from "next/link"
import { Button } from "@/components/ui/button"
import WorkoutForm from "@/components/gym/workout-form"
import WorkoutList from "@/components/gym/workout-list"
import RoutineList from "@/components/gym/routine-list"
import RoutineForm from "@/components/gym/routine-form"
import RoutineDetail from "@/components/gym/routine-detail"
import ExerciseHistory from "@/components/gym/exercise-history"
import GymMetrics from "@/components/gym/gym-metrics"

// Interfaces TypeScript
interface Workout {
  id: string
  exercise_name: string
  weight_kg: number | null
  repetitions: number | null
  sets: number | null
  created_at: string
}

type ViewMode = "routines" | "individual" | "routine-detail" | "create-routine" | "history" | "metrics"

export default function GymPage() {
  // Estados del componente
  const [refreshTrigger, setRefreshTrigger] = useState(0)
  const [editingWorkout, setEditingWorkout] = useState<Workout | null>(null)
  const [viewMode, setViewMode] = useState<ViewMode>("routines")
  const [selectedRoutine, setSelectedRoutine] = useState<{ id: string; name: string } | null>(null)

  // Handlers para diferentes acciones
  const handleWorkoutAdded = () => {
    setRefreshTrigger((prev) => prev + 1)
  }

  const handleEditWorkout = (workout: Workout) => {
    setEditingWorkout(workout)
  }

  const handleEditComplete = () => {
    setEditingWorkout(null)
    setRefreshTrigger((prev) => prev + 1)
  }

  const handleViewRoutine = (routineId: string, routineName: string) => {
    setSelectedRoutine({ id: routineId, name: routineName })
    setViewMode("routine-detail")
  }

  const handleCreateRoutine = () => {
    setViewMode("create-routine")
  }

  const handleRoutineCreated = () => {
    setViewMode("routines")
    setRefreshTrigger((prev) => prev + 1)
  }

  const handleBackToRoutines = () => {
    setSelectedRoutine(null)
    setViewMode("routines")
    setRefreshTrigger((prev) => prev + 1)
  }

  return (
    <div className="min-h-screen bg-gradient-to-br from-blue-50 to-indigo-100 dark:from-gray-900 dark:to-gray-800">
      <div className="container mx-auto px-4 py-8 max-w-4xl">
        {/* Header con navegacion */}
        <div className="mb-8">
          <div className="flex items-center gap-4 mb-4">
            <Button variant="outline" size="sm" asChild>
              <Link href="/">
                <ArrowLeft className="h-4 w-4 mr-2" />
                Volver
              </Link>
            </Button>
          </div>
          <div className="flex items-center gap-3 mb-2">
            <Dumbbell className="h-8 w-8 text-blue-600 dark:text-blue-400" />
            <h1 className="text-3xl font-bold text-gray-900 dark:text-white">Gimnasio</h1>
          </div>
          <p className="text-gray-600 dark:text-gray-300">
            Organiza tus entrenamientos por rutinas o registra ejercicios individuales
          </p>
        </div>

        {/* Pestanas de navegacion */}
        <div className="flex gap-2 mb-6 flex-wrap">
          <button
            onClick={() => setViewMode("routines")}
            className={`px-4 py-2 rounded-lg font-medium transition-colors ${
              viewMode === "routines" || viewMode === "routine-detail" || viewMode === "create-routine"
                ? "bg-blue-600 text-white"
                : "bg-gray-100 dark:bg-gray-700 text-gray-700 dark:text-gray-300 hover:bg-gray-200 dark:hover:bg-gray-600"
            }`}
          >
            Rutinas
          </button>
          <button
            onClick={() => setViewMode("individual")}
            className={`px-4 py-2 rounded-lg font-medium transition-colors ${
              viewMode === "individual" ? "bg-blue-600 text-white" : "bg-gray-100 dark:bg-gray-700 text-gray-700 dark:text-gray-300 hover:bg-gray-200 dark:hover:bg-gray-600"
            }`}
          >
            Ejercicios Individuales
          </button>
          <button
            onClick={() => setViewMode("history")}
            className={`px-4 py-2 rounded-lg font-medium transition-colors ${
              viewMode === "history" ? "bg-blue-600 text-white" : "bg-gray-100 dark:bg-gray-700 text-gray-700 dark:text-gray-300 hover:bg-gray-200 dark:hover:bg-gray-600"
            }`}
          >
            Historial
          </button>
          <button
            onClick={() => setViewMode("metrics")}
            className={`px-4 py-2 rounded-lg font-medium transition-colors ${
              viewMode === "metrics" ? "bg-blue-600 text-white" : "bg-gray-100 dark:bg-gray-700 text-gray-700 dark:text-gray-300 hover:bg-gray-200 dark:hover:bg-gray-600"
            }`}
          >
            Metricas
          </button>
        </div>

        {/* Renderizado condicional de componentes */}
        <div className="grid gap-6">
          {viewMode === "routines" && (
            <RoutineList
              refreshTrigger={refreshTrigger}
              onViewRoutine={handleViewRoutine}
              onCreateRoutine={handleCreateRoutine}
            />
          )}

          {viewMode === "create-routine" && (
            <RoutineForm 
              onRoutineCreated={handleRoutineCreated} 
              onCancel={() => setViewMode("routines")} 
            />
          )}

          {viewMode === "routine-detail" && selectedRoutine && (
            <RoutineDetail
              routineId={selectedRoutine.id}
              routineName={selectedRoutine.name}
              onBack={handleBackToRoutines}
            />
          )}

          {viewMode === "individual" && (
            <>
              <WorkoutForm
                onWorkoutAdded={handleWorkoutAdded}
                editWorkout={editingWorkout}
                onEditComplete={handleEditComplete}
              />
              <WorkoutList 
                refreshTrigger={refreshTrigger} 
                onEditWorkout={handleEditWorkout} 
              />
            </>
          )}

          {viewMode === "history" && <ExerciseHistory />}
          {viewMode === "metrics" && <GymMetrics />}
        </div>
      </div>
    </div>
  )
}
\end{lstlisting}

\textbf{Caracteristicas del componente:}
\begin{itemize}
    \item \textbf{Estado centralizado}: Maneja todos los estados de la aplicacion
    \item \textbf{Navegacion por pestanas}: Interfaz intuitiva para cambiar entre vistas
    \item \textbf{Renderizado condicional}: Muestra diferentes componentes segun el modo
    \item \textbf{Handlers de eventos}: Gestiona todas las interacciones del usuario
    \item \textbf{TypeScript}: Tipado estricto para mayor seguridad
\end{itemize}

\section{Ejemplos Practicos de Uso}

\subsection{Ejemplos de Insercion a Base de Datos}

\subsubsection{Crear un Entrenamiento Individual}

\begin{lstlisting}[style=sqlstyle, caption=Ejemplo de insercion en gym_workouts]
-- Insertar un entrenamiento de press de banca
INSERT INTO public.gym_workouts (
    user_id,
    exercise_name,
    weight_kg,
    repetitions,
    sets,
    image_url
) VALUES (
    '123e4567-e89b-12d3-a456-426614174000', -- UUID del usuario
    'Press de Banca',
    80.00,
    12,
    3,
    'https://ejemplo.com/press-banca.jpg'
);

-- Insertar un entrenamiento de sentadillas
INSERT INTO public.gym_workouts (
    user_id,
    exercise_name,
    weight_kg,
    repetitions,
    sets
) VALUES (
    '123e4567-e89b-12d3-a456-426614174000',
    'Sentadillas',
    100.00,
    10,
    4
);
\end{lstlisting}

\subsubsection{Crear una Rutina Completa}

\begin{lstlisting}[style=sqlstyle, caption=Ejemplo de creacion de rutina completa]
-- 1. Crear la rutina
INSERT INTO public.routines (
    user_id,
    name,
    description
) VALUES (
    '123e4567-e89b-12d3-a456-426614174000',
    'Rutina de Pecho y Triceps',
    'Rutina enfocada en el desarrollo del pecho y triceps, ideal para principiantes'
);

-- Obtener el ID de la rutina recien creada
-- (En la aplicacion esto se maneja automaticamente)

-- 2. Agregar ejercicios a la rutina
INSERT INTO public.routine_exercises (
    routine_id,
    exercise_name,
    weight,
    repetitions,
    sets,
    order_index
) VALUES 
    ('456e7890-e89b-12d3-a456-426614174001', 'Press de Banca', 80.00, 12, 3, 0),
    ('456e7890-e89b-12d3-a456-426614174001', 'Aperturas con Mancuernas', 25.00, 15, 3, 1),
    ('456e7890-e89b-12d3-a456-426614174001', 'Press Frances', 30.00, 12, 3, 2),
    ('456e7890-e89b-12d3-a456-426614174001', 'Fondos para Triceps', 0.00, 10, 3, 3);
\end{lstlisting}

\subsubsection{Consultas Utiles}

\begin{lstlisting}[style=sqlstyle, caption=Consultas utiles para analisis]
-- Obtener todos los entrenamientos de un usuario con detalles
SELECT 
    gw.exercise_name,
    gw.weight_kg,
    gw.repetitions,
    gw.sets,
    gw.created_at,
    (gw.weight_kg * gw.repetitions * gw.sets) as volumen_total
FROM public.gym_workouts gw
WHERE gw.user_id = '123e4567-e89b-12d3-a456-426614174000'
ORDER BY gw.created_at DESC;

-- Obtener progreso de un ejercicio especifico
SELECT 
    exercise_name,
    weight_kg,
    repetitions,
    sets,
    created_at,
    (weight_kg * repetitions * sets) as volumen
FROM public.gym_workouts
WHERE user_id = '123e4567-e89b-12d3-a456-426614174000'
  AND exercise_name = 'Press de Banca'
ORDER BY created_at ASC;

-- Obtener rutinas con conteo de ejercicios
SELECT 
    r.name,
    r.description,
    r.created_at,
    COUNT(re.id) as total_ejercicios
FROM public.routines r
LEFT JOIN public.routine_exercises re ON r.id = re.routine_id
WHERE r.user_id = '123e4567-e89b-12d3-a456-426614174000'
GROUP BY r.id, r.name, r.description, r.created_at
ORDER BY r.created_at DESC;

-- Obtener ejercicios de una rutina especifica en orden
SELECT 
    re.exercise_name,
    re.weight,
    re.repetitions,
    re.sets,
    re.order_index
FROM public.routine_exercises re
WHERE re.routine_id = '456e7890-e89b-12d3-a456-426614174001'
ORDER BY re.order_index ASC;
\end{lstlisting}

\subsection{Flujos de Datos Completos}

\subsubsection{Flujo de Creacion de Entrenamiento}

\begin{enumerate}
    \item \textbf{Usuario completa formulario} en \texttt{WorkoutForm}
    \item \textbf{Validacion en cliente} de campos requeridos
    \item \textbf{Envio a Server Action} \texttt{createWorkout}
    \item \textbf{Verificacion de autenticacion} en servidor
    \item \textbf{Sanitizacion de datos} (conversion de tipos)
    \item \textbf{Insercion en base de datos} tabla \texttt{gym\_workouts}
    \item \textbf{Revalidacion de cache} con \texttt{revalidatePath}
    \item \textbf{Actualizacion de UI} con nuevo entrenamiento
\end{enumerate}

\subsubsection{Flujo de Gestion de Rutinas}

\begin{enumerate}
    \item \textbf{Creacion de rutina} con \texttt{createRoutine}
    \item \textbf{Insercion en tabla} \texttt{routines}
    \item \textbf{Agregar ejercicios} con \texttt{addExerciseToRoutine}
    \item \textbf{Insercion en tabla} \texttt{routine\_exercises} con \texttt{order\_index}
    \item \textbf{Visualizacion de rutina} con \texttt{getRoutineExercises}
    \item \textbf{Ejecucion de rutina} (registro de entrenamientos individuales)
\end{enumerate}

\section{Caracteristicas Avanzadas}

\subsection{Selector de Ejercicios}

El modulo incluye un selector modal avanzado que permite:

\begin{itemize}
    \item \textbf{Catalogo de ejercicios}: Acceso a ejercicios administrados
    \item \textbf{Filtrado por categoria}: Pecho, Biceps, Triceps, etc.
    \item \textbf{Busqueda de ejercicios}: Filtrado por nombre
    \item \textbf{Ejercicios personalizados}: Creacion de ejercicios unicos
    \item \textbf{Imagenes de ejercicios}: Visualizacion de tecnicas
\end{itemize}

\subsection{Metricas y Estadisticas}

El componente \texttt{GymMetrics} proporciona:

\begin{itemize}
    \item \textbf{Graficos de progreso}: Evolucion del peso a lo largo del tiempo
    \item \textbf{Estadisticas de volumen}: Calculo de volumen total (peso × reps × series)
    \item \textbf{Tendencias de entrenamiento}: Frecuencia y consistencia
    \item \textbf{Comparativas temporales}: Progreso mensual/semanal
    \item \textbf{Analisis por ejercicio}: Progreso especifico por ejercicio
\end{itemize}

\subsection{Historial de Ejercicios}

El componente \texttt{ExerciseHistory} permite:

\begin{itemize}
    \item \textbf{Historial completo}: Todos los entrenamientos registrados
    \item \textbf{Filtrado por ejercicio}: Ver progreso de un ejercicio especifico
    \item \textbf{Analisis temporal}: Progreso a lo largo del tiempo
    \item \textbf{Exportacion de datos}: Para analisis externo
    \item \textbf{Busqueda avanzada}: Filtros por fecha, ejercicio, peso
\end{itemize}

\section{Mejores Practicas de Desarrollo}

\subsection{Seguridad}

\begin{itemize}
    \item \textbf{Validacion de entrada}: Siempre validar datos en Server Actions
    \item \textbf{Autenticacion}: Verificar usuario en cada operacion
    \item \textbf{RLS}: Usar Row Level Security en todas las tablas
    \item \textbf{Sanitizacion}: Limpiar y convertir datos de entrada
    \item \textbf{Logs de seguridad}: Registrar operaciones sensibles
\end{itemize}

\subsection{Performance}

\begin{itemize}
    \item \textbf{Indices de base de datos}: Optimizar consultas frecuentes
    \item \textbf{Paginacion}: Implementar para listas largas
    \item \textbf{Cache}: Usar revalidatePath para actualizar cache
    \item \textbf{Lazy loading}: Cargar componentes bajo demanda
    \item \textbf{Optimizacion de imagenes}: Comprimir y optimizar imagenes
\end{itemize}

\subsection{Mantenibilidad}

\begin{itemize}
    \item \textbf{TypeScript}: Usar tipado estricto en todos los componentes
    \item \textbf{Interfaces}: Definir interfaces claras para props
    \item \textbf{Separacion de responsabilidades}: Logica en Server Actions
    \item \textbf{Reutilizacion}: Componentes modulares y reutilizables
    \item \textbf{Documentacion}: Comentar codigo complejo
\end{itemize}

\section{Solucion de Problemas}

\subsection{Problemas Comunes}

\subsubsection{Error: Tabla no existe}

\textbf{Sintomas}: Error al intentar insertar o consultar datos
\textbf{Causa}: Las tablas no han sido creadas en la base de datos
\textbf{Solucion}: Ejecutar los scripts SQL en orden:
\begin{enumerate}
    \item \texttt{01-create-database-schema.sql}
    \item \texttt{01-create-user-schema.sql}
    \item \texttt{08-verify-routines-tables.sql}
    \item \texttt{26-create-gym-exercises-table.sql}
\end{enumerate}

\subsubsection{Error: Usuario no autenticado}

\textbf{Sintomas}: Server Actions retornan error de autenticacion
\textbf{Causa}: Usuario no esta logueado o sesion expirada
\textbf{Solucion}: Verificar estado de autenticacion y redirigir a login

\subsubsection{Error: Politica RLS violada}

\textbf{Sintomas}: Error al acceder a datos de otros usuarios
\textbf{Causa}: Politicas RLS mal configuradas
\textbf{Solucion}: Verificar y corregir politicas RLS

\subsection{Debugging}

\subsubsection{Logs del Cliente}

\begin{lstlisting}[caption=Debugging en cliente]
// Habilitar logs detallados
localStorage.setItem('debug', 'true');

// Verificar estado de autenticacion
console.log('User:', user);
console.log('Session:', session);

// Verificar datos de entrenamientos
console.log('Workouts:', workouts);
\end{lstlisting}

\subsubsection{Logs del Servidor}

\begin{lstlisting}[caption=Debugging en servidor]
// En Server Actions
console.log('Action called with:', { userId, data });

// En consultas de base de datos
console.log('Query result:', { data, error });

// En validaciones
console.log('Validation result:', validationResult);
\end{lstlisting}

\section{Conclusion}

El modulo de gimnasio de FitTrack es un sistema completo y robusto que permite a los usuarios gestionar sus entrenamientos de manera eficiente. Con su arquitectura bien definida, base de datos optimizada y componentes React modernos, proporciona una experiencia de usuario excepcional.

\textbf{Caracteristicas destacadas:}
\begin{itemize}
    \item Arquitectura escalable y mantenible
    \item Seguridad robusta con RLS
    \item Interfaz de usuario intuitiva
    \item Funcionalidades avanzadas de analisis
    \item Codigo bien documentado y tipado
\end{itemize}

Para contribuir al desarrollo del modulo:
\begin{enumerate}
    \item Seguir las convenciones establecidas
    \item Implementar tests apropiados
    \item Documentar cambios significativos
    \item Mantener la compatibilidad con la base de datos
    \item Respetar las politicas de seguridad
\end{enumerate}

\end{document}
