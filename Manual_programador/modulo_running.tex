\documentclass[12pt,a4paper]{article}
\usepackage[utf8]{inputenc}
\usepackage[spanish]{babel}
\usepackage{geometry}
\usepackage{graphicx}
\usepackage{hyperref}
\usepackage{listings}
\usepackage{xcolor}
\usepackage{booktabs}
\usepackage{array}
\usepackage{longtable}
\usepackage{fancyhdr}
\usepackage{titlesec}
\usepackage{enumitem}
\usepackage{amsmath}
\usepackage{amsfonts}
\usepackage{amssymb}
\usepackage{float}

% Configuración de página
\geometry{margin=2.5cm}
\pagestyle{fancy}
\fancyhf{}
\fancyhead[L]{Modulo de Running - FitTrack}
\fancyhead[R]{\thepage}
\renewcommand{\headrulewidth}{0.4pt}
\setlength{\headheight}{15pt}

% Configuración de colores para código
\definecolor{codegreen}{rgb}{0,0.6,0}
\definecolor{codegray}{rgb}{0.5,0.5,0.5}
\definecolor{codepurple}{rgb}{0.58,0,0.82}
\definecolor{backcolour}{rgb}{0.95,0.95,0.92}
\definecolor{bluekeywords}{rgb}{0.13,0.13,1}
\definecolor{greencomments}{rgb}{0,0.5,0}
\definecolor{redstrings}{rgb}{0.9,0,0}

% Configuración de listings
\lstdefinestyle{mystyle}{
    backgroundcolor=\color{backcolour},   
    commentstyle=\color{greencomments},
    keywordstyle=\color{bluekeywords},
    numberstyle=\tiny\color{codegray},
    stringstyle=\color{redstrings},
    basicstyle=\ttfamily\footnotesize,
    breakatwhitespace=false,         
    breaklines=true,                 
    captionpos=b,                    
    keepspaces=true,                 
    numbers=left,                    
    numbersep=5pt,                  
    showspaces=false,                
    showstringspaces=false,
    showtabs=false,                  
    tabsize=2,
    frame=single,
    rulecolor=\color{codegray}
}

\lstdefinestyle{sqlstyle}{
    backgroundcolor=\color{backcolour},   
    commentstyle=\color{greencomments},
    keywordstyle=\color{bluekeywords},
    numberstyle=\tiny\color{codegray},
    stringstyle=\color{redstrings},
    basicstyle=\ttfamily\footnotesize,
    breakatwhitespace=false,         
    breaklines=true,                 
    captionpos=b,                    
    keepspaces=true,                 
    numbers=left,                    
    numbersep=5pt,                  
    showspaces=false,                
    showstringspaces=false,
    showtabs=false,                  
    tabsize=2,
    frame=single,
    rulecolor=\color{codegray}
}

\lstset{style=mystyle}

% Configuración de títulos
\titleformat{\section}{\Large\bfseries}{\thesection}{1em}{}
\titleformat{\subsection}{\large\bfseries}{\thesubsection}{1em}{}
\titleformat{\subsubsection}{\normalsize\bfseries}{\thesubsubsection}{1em}{}

% Configuración de hipervínculos
\hypersetup{
    colorlinks=true,
    linkcolor=blue,
    filecolor=magenta,      
    urlcolor=cyan,
    citecolor=red,
}

\begin{document}

% Portada
\begin{titlepage}
\centering
\vspace*{2cm}

{\Huge\bfseries Modulo de Running}\\[0.5cm]
{\LARGE FitTrack}\\[1cm]

{\large Documentacion Tecnica Completa}\\[2cm]

\begin{minipage}{0.8\textwidth}
\centering
Este documento proporciona una guia tecnica exhaustiva del modulo de running de FitTrack, incluyendo arquitectura, implementacion, base de datos, componentes React, Server Actions y ejemplos practicos de uso.
\end{minipage}

\vfill

{\large Version 1.0}\\[0.5cm]
{\large \today}

\end{titlepage}

\tableofcontents
\newpage

\section{Introduccion al Modulo de Running}

El modulo de running es un componente fundamental de FitTrack que permite a los usuarios registrar, analizar y hacer seguimiento de sus sesiones de carrera. Proporciona herramientas completas para el monitoreo del progreso, calculo automatico de metricas y visualizacion de estadisticas detalladas.

\subsection{Caracteristicas Principales}

\begin{itemize}
    \item \textbf{Registro de Sesiones}: Permite registrar sesiones de carrera con duracion, distancia y ritmo
    \item \textbf{Calculo Automatico de Ritmo}: Calcula automaticamente el ritmo por kilometro basado en duracion y distancia
    \item \textbf{Estadisticas Detalladas}: Proporciona metricas completas como total de sesiones, distancia acumulada, tiempo total, ritmo promedio y mejor ritmo
    \item \textbf{Historial de Sesiones}: Mantiene un registro completo de todas las sesiones de carrera
    \item \textbf{Graficos de Progreso}: Visualizacion de tendencias y progreso a lo largo del tiempo
    \item \textbf{Analisis de Rendimiento}: Comparacion de sesiones y identificacion de mejoras
    \item \textbf{Interfaz Intuitiva}: Diseño limpio y facil de usar para registro rapido
\end{itemize}

\subsection{Arquitectura General}

El modulo sigue una arquitectura de capas bien definida:

\begin{enumerate}
    \item \textbf{Capa de Presentacion}: Componentes React con TypeScript
    \item \textbf{Capa de Logica}: Server Actions de Next.js
    \item \textbf{Capa de Datos}: Supabase PostgreSQL con RLS
    \item \textbf{Capa de Servicios}: Calculos y utilidades de metricas
\end{enumerate}

\section{Estructura de Base de Datos}

\subsection{Diagrama de Relaciones}

\begin{figure}[H]
\centering
\begin{verbatim}
    auth.users
         |
    +----+----+
    |         |
running_sessions
    |
running_history (opcional)
\end{verbatim}
\caption{Diagrama de relaciones de las tablas del modulo de running}
\end{figure}

\subsection{Tabla running\_sessions}

Esta tabla almacena las sesiones de carrera realizadas por los usuarios.

\begin{lstlisting}[style=sqlstyle, caption=Estructura completa de running_sessions]
CREATE TABLE IF NOT EXISTS public.running_sessions (
    id UUID DEFAULT gen_random_uuid() PRIMARY KEY,
    user_id UUID REFERENCES auth.users(id) ON DELETE CASCADE NOT NULL,
    duration_minutes INTEGER NOT NULL CHECK (duration_minutes > 0),
    distance_km NUMERIC(5,2) NOT NULL CHECK (distance_km > 0),
    pace_min_km NUMERIC(5,2),
    created_at TIMESTAMP WITH TIME ZONE DEFAULT NOW()
);

-- Indices para optimizacion
CREATE INDEX IF NOT EXISTS idx_running_sessions_user_id ON public.running_sessions(user_id);
CREATE INDEX IF NOT EXISTS idx_running_sessions_created_at ON public.running_sessions(created_at);
CREATE INDEX IF NOT EXISTS idx_running_sessions_distance ON public.running_sessions(distance_km);
CREATE INDEX IF NOT EXISTS idx_running_sessions_pace ON public.running_sessions(pace_min_km);
\end{lstlisting}

\textbf{Descripcion de campos:}
\begin{itemize}
    \item \texttt{id}: Identificador unico UUID generado automaticamente
    \item \texttt{user\_id}: Referencia al usuario propietario de la sesion
    \item \texttt{duration\_minutes}: Duracion de la sesion en minutos (requerido, > 0)
    \item \texttt{distance\_km}: Distancia recorrida en kilometros (requerido, > 0)
    \item \texttt{pace\_min\_km}: Ritmo en minutos por kilometro (opcional, calculado automaticamente)
    \item \texttt{created\_at}: Timestamp de creacion automatico
\end{itemize}

\subsection{Tabla running\_history (Opcional)}

Tabla adicional para historial detallado de sesiones de running.

\begin{lstlisting}[style=sqlstyle, caption=Estructura de running_history]
CREATE TABLE IF NOT EXISTS running_history (
    id UUID DEFAULT gen_random_uuid() PRIMARY KEY,
    user_id UUID REFERENCES auth.users(id) ON DELETE CASCADE NOT NULL,
    duration INTEGER NOT NULL CHECK (duration > 0),
    distance DECIMAL(6,2) NOT NULL CHECK (distance > 0),
    pace DECIMAL(4,2) NOT NULL CHECK (pace > 0),
    created_at TIMESTAMP WITH TIME ZONE DEFAULT NOW(),
    updated_at TIMESTAMP WITH TIME ZONE DEFAULT NOW()
);

-- Indices para optimizacion
CREATE INDEX IF NOT EXISTS idx_running_history_user_id ON running_history(user_id);
CREATE INDEX IF NOT EXISTS idx_running_history_created_at ON running_history(created_at);
\end{lstlisting}

\subsection{Politicas de Seguridad (RLS)}

Todas las tablas implementan Row Level Security para garantizar que los usuarios solo accedan a sus propios datos.

\begin{lstlisting}[style=sqlstyle, caption=Politicas RLS para running_sessions]
-- Habilitar RLS
ALTER TABLE public.running_sessions ENABLE ROW LEVEL SECURITY;

-- Politicas para running_sessions
CREATE POLICY "Users can view own running sessions" 
ON public.running_sessions
FOR SELECT USING (auth.uid() = user_id);

CREATE POLICY "Users can insert own running sessions" 
ON public.running_sessions
FOR INSERT WITH CHECK (auth.uid() = user_id);

CREATE POLICY "Users can update own running sessions" 
ON public.running_sessions
FOR UPDATE USING (auth.uid() = user_id);

CREATE POLICY "Users can delete own running sessions" 
ON public.running_sessions
FOR DELETE USING (auth.uid() = user_id);
\end{lstlisting}

\section{Server Actions - Logica de Negocio}

Las Server Actions manejan toda la logica de negocio del modulo de running. Estan implementadas en TypeScript con validacion robusta y manejo de errores.

\subsection{Archivo running-actions.ts}

\subsubsection{createRunningSession - Crear Sesion de Running}

\begin{lstlisting}[caption=Funcion createRunningSession completa]
"use server"

import { revalidatePath } from "next/cache"
import { createClient } from "@/lib/supabase/server"

export async function createRunningSession(prevState: any, formData: FormData) {
  // Extraer datos del formulario
  const duration_minutes = formData.get("duration_minutes")?.toString()
  const distance_km = formData.get("distance_km")?.toString()
  const pace_min_km = formData.get("pace_min_km")?.toString()

  // Validacion basica
  if (!duration_minutes || !distance_km) {
    return { error: "Duracion y distancia son requeridos" }
  }

  const durationNum = Number.parseInt(duration_minutes)
  const distanceNum = Number.parseFloat(distance_km)

  if (durationNum <= 0 || distanceNum <= 0) {
    return { error: "Duracion y distancia deben ser mayores a 0" }
  }

  // Calcular ritmo si no se proporciona
  let calculatedPace: number | null = null
  if (pace_min_km && pace_min_km.trim() !== "") {
    calculatedPace = Number.parseFloat(pace_min_km)
  } else {
    calculatedPace = durationNum / distanceNum
  }

  // Verificar autenticacion
  const supabase = await createClient()
  const {
    data: { user },
  } = await supabase.auth.getUser()

  if (!user) {
    return { error: "Usuario no autenticado" }
  }

  try {
    // Preparar datos para insercion
    const insertData = {
      user_id: user.id,
      duration_minutes: durationNum,
      distance_km: distanceNum,
      pace_min_km: calculatedPace,
    }

    // Insertar en base de datos
    const { error } = await supabase
      .from("running_sessions")
      .insert(insertData)

    if (error) {
      console.error("Database error:", error)
      return { error: "Error al guardar la sesion" }
    }

    // Revalidar cache
    revalidatePath("/running")
    return { success: true }
  } catch (error) {
    console.error("Error:", error)
    return { error: "Error al guardar la sesion" }
  }
}
\end{lstlisting}

\textbf{Caracteristicas importantes:}
\begin{itemize}
    \item \textbf{Validacion de entrada}: Verifica que duracion y distancia esten presentes y sean validos
    \item \textbf{Calculo automatico de ritmo}: Calcula el ritmo si no se proporciona (duracion / distancia)
    \item \textbf{Autenticacion}: Confirma que el usuario este autenticado
    \item \textbf{Sanitizacion}: Convierte y valida valores numericos
    \item \textbf{Manejo de errores}: Captura y reporta errores de base de datos
    \item \textbf{Revalidacion}: Actualiza el cache de Next.js
\end{itemize}

\subsubsection{getRunningSessions - Obtener Sesiones}

\begin{lstlisting}[caption=Funcion getRunningSessions completa]
export async function getRunningSessions() {
  const supabase = await createClient()
  const {
    data: { user },
  } = await supabase.auth.getUser()

  if (!user) {
    return []
  }

  try {
    const { data, error } = await supabase
      .from("running_sessions")
      .select("*")
      .eq("user_id", user.id)
      .order("created_at", { ascending: false })

    if (error) {
      console.error("Database error:", error)
      return []
    }

    return data || []
  } catch (error) {
    console.error("Error:", error)
    return []
  }
}
\end{lstlisting}

\subsubsection{deleteRunningSession - Eliminar Sesion}

\begin{lstlisting}[caption=Funcion deleteRunningSession completa]
export async function deleteRunningSession(sessionId: string) {
  const supabase = await createClient()
  const {
    data: { user },
  } = await supabase.auth.getUser()

  if (!user) {
    return { error: "Usuario no autenticado" }
  }

  try {
    const { error } = await supabase
      .from("running_sessions")
      .delete()
      .eq("id", sessionId)
      .eq("user_id", user.id)

    if (error) {
      console.error("Database error:", error)
      return { error: "Error al eliminar la sesion" }
    }

    revalidatePath("/running")
    return { success: true }
  } catch (error) {
    console.error("Error:", error)
    return { error: "Error al eliminar la sesion" }
  }
}
\end{lstlisting}

\section{Componentes React}

\subsection{Pagina Principal - RunningPage}

El componente principal que coordina toda la funcionalidad del modulo de running.

\begin{lstlisting}[caption=app/running/page.tsx - Estructura completa]
"use client"

import { useState } from "react"
import { MapPin, ArrowLeft } from "lucide-react"
import Link from "next/link"
import { Button } from "@/components/ui/button"
import RunningForm from "@/components/running/running-form"
import RunningList from "@/components/running/running-list"
import RunningStats from "@/components/running/running-stats"
import RunningCharts from "@/components/running/running-charts"

export default function RunningPage() {
  // Estados del componente
  const [refreshTrigger, setRefreshTrigger] = useState(0)
  const [viewMode, setViewMode] = useState<"sessions" | "charts">("sessions")

  // Handler para cuando se agrega una sesion
  const handleSessionAdded = () => {
    setRefreshTrigger((prev) => prev + 1)
  }

  return (
    <div className="container mx-auto px-4 py-8 max-w-4xl">
      {/* Header con navegacion */}
      <div className="mb-8">
        <div className="flex items-center gap-4 mb-4">
          <Button variant="outline" size="sm" asChild>
            <Link href="/">
              <ArrowLeft className="h-4 w-4 mr-2" />
              Volver
            </Link>
          </Button>
        </div>
        <div className="flex items-center gap-3 mb-2">
          <MapPin className="h-8 w-8 text-green-600" />
          <h1 className="text-3xl font-bold">Running</h1>
        </div>
        <p className="text-gray-600">Registra y analiza tus sesiones de carrera</p>
      </div>

      {/* Pestanas de navegacion */}
      <div className="flex gap-2 mb-6">
        <button
          onClick={() => setViewMode("sessions")}
          className={`px-4 py-2 rounded-lg font-medium transition-colors ${
            viewMode === "sessions" ? "bg-green-600 text-white" : "bg-gray-100 text-gray-700 hover:bg-gray-200"
          }`}
        >
          Sesiones
        </button>
        <button
          onClick={() => setViewMode("charts")}
          className={`px-4 py-2 rounded-lg font-medium transition-colors ${
            viewMode === "charts" ? "bg-green-600 text-white" : "bg-gray-100 text-gray-700 hover:bg-gray-200"
          }`}
        >
          Graficos
        </button>
      </div>

      {/* Renderizado condicional de componentes */}
      <div className="grid gap-6">
        {viewMode === "sessions" && (
          <>
            <RunningForm onSessionAdded={handleSessionAdded} />
            <RunningStats refreshTrigger={refreshTrigger} />
            <RunningList refreshTrigger={refreshTrigger} />
          </>
        )}

        {viewMode === "charts" && <RunningCharts />}
      </div>
    </div>
  )
}
\end{lstlisting}

\textbf{Caracteristicas del componente:}
\begin{itemize}
    \item \textbf{Estado centralizado}: Maneja el estado de actualizacion y modo de vista
    \item \textbf{Navegacion por pestanas}: Interfaz intuitiva para cambiar entre sesiones y graficos
    \item \textbf{Renderizado condicional}: Muestra diferentes componentes segun el modo
    \item \textbf{Handlers de eventos}: Gestiona las interacciones del usuario
    \item \textbf{TypeScript}: Tipado estricto para mayor seguridad
\end{itemize}

\subsection{Componente RunningForm}

Formulario para registrar nuevas sesiones de running.

\begin{lstlisting}[caption=components/running/running-form.tsx - Estructura principal]
"use client"

import { useState } from "react"
import { Button } from "@/components/ui/button"
import { Input } from "@/components/ui/input"
import { Card, CardContent, CardDescription, CardHeader, CardTitle } from "@/components/ui/card"
import { Label } from "@/components/ui/label"
import { Plus, Loader2 } from "lucide-react"
import { createRunningSession } from "@/lib/running-actions"
import { useActionState } from "react"

export default function RunningForm({ onSessionAdded }: { onSessionAdded?: () => void }) {
  const [state, formAction] = useActionState(createRunningSession, null)
  const [isOpen, setIsOpen] = useState(false)

  const handleSubmit = async (formData: FormData) => {
    const result = await formAction(formData)
    if (result?.success) {
      setIsOpen(false)
      onSessionAdded?.()
    }
  }

  // Renderizar boton si no esta abierto
  if (!isOpen) {
    return (
      <Button onClick={() => setIsOpen(true)} className="w-full bg-green-600 hover:bg-green-700">
        <Plus className="h-4 w-4 mr-2" />
        Registrar Sesion de Running
      </Button>
    )
  }

  return (
    <Card>
      <CardHeader>
        <CardTitle>Nueva Sesion de Running</CardTitle>
        <CardDescription>Registra tu sesion de carrera</CardDescription>
      </CardHeader>
      <CardContent>
        <form action={handleSubmit} className="space-y-4">
          {/* Mostrar errores si existen */}
          {state?.error && (
            <div className="bg-red-50 border border-red-200 text-red-700 px-4 py-3 rounded text-sm">
              {state.error}
            </div>
          )}

          <div className="grid grid-cols-1 gap-4">
            <div className="grid grid-cols-2 gap-4">
              <div className="space-y-2">
                <Label htmlFor="duration_minutes">Duracion (minutos)</Label>
                <Input
                  id="duration_minutes"
                  name="duration_minutes"
                  type="number"
                  min="1"
                  step="1"
                  placeholder="30"
                  required
                />
              </div>
              <div className="space-y-2">
                <Label htmlFor="distance_km">Distancia (km)</Label>
                <Input
                  id="distance_km"
                  name="distance_km"
                  type="number"
                  min="0.1"
                  step="0.1"
                  placeholder="5.0"
                  required
                />
              </div>
            </div>

            <div className="space-y-2">
              <Label htmlFor="pace_min_km">Ritmo (min/km) - Opcional</Label>
              <Input 
                id="pace_min_km" 
                name="pace_min_km" 
                type="number" 
                min="0" 
                step="0.1" 
                placeholder="6.0" 
              />
              <p className="text-xs text-gray-500">
                Se calculara automaticamente si no se especifica
              </p>
            </div>
          </div>

          {/* Botones de accion */}
          <div className="flex gap-2">
            <Button type="submit" className="bg-green-600 hover:bg-green-700">
              {state?.success === false ? (
                <Loader2 className="h-4 w-4 mr-2 animate-spin" />
              ) : null}
              Guardar Sesion
            </Button>
            <Button 
              type="button" 
              variant="outline" 
              onClick={() => setIsOpen(false)}
            >
              Cancelar
            </Button>
          </div>
        </form>
      </CardContent>
    </Card>
  )
}
\end{lstlisting}

\textbf{Caracteristicas del formulario:}
\begin{itemize}
    \item \textbf{Estado colapsable}: Se muestra como boton inicialmente, se expande al hacer clic
    \item \textbf{Validacion en cliente}: Campos requeridos y validacion de tipos
    \item \textbf{Calculo automatico}: El ritmo se calcula automaticamente si no se proporciona
    \item \textbf{Manejo de errores}: Muestra errores de validacion y base de datos
    \item \textbf{Feedback visual}: Indicador de carga durante el envio
\end{itemize}

\subsection{Componente RunningStats}

Componente que calcula y muestra estadisticas detalladas de las sesiones de running.

\begin{lstlisting}[caption=components/running/running-stats.tsx - Logica de calculo]
interface Stats {
  totalSessions: number
  totalDistance: number
  totalTime: number
  averagePace: number | null
  bestPace: number | null
  longestRun: number
}

export default function RunningStats({ refreshTrigger }: { refreshTrigger?: number }) {
  const [stats, setStats] = useState<Stats | null>(null)
  const [loading, setLoading] = useState(true)

  const calculateStats = (sessions: RunningSession[]): Stats => {
    if (sessions.length === 0) {
      return {
        totalSessions: 0,
        totalDistance: 0,
        totalTime: 0,
        averagePace: null,
        bestPace: null,
        longestRun: 0,
      }
    }

    // Calcular metricas basicas
    const totalDistance = sessions.reduce((sum, session) => sum + session.distance_km, 0)
    const totalTime = sessions.reduce((sum, session) => sum + session.duration_minutes, 0)
    const longestRun = Math.max(...sessions.map((session) => session.distance_km))

    // Calcular ritmo promedio y mejor ritmo
    const sessionsWithPace = sessions.filter((session) => session.pace_min_km !== null)
    const averagePace =
      sessionsWithPace.length > 0
        ? sessionsWithPace.reduce((sum, session) => sum + (session.pace_min_km || 0), 0) / sessionsWithPace.length
        : null

    const bestPace =
      sessionsWithPace.length > 0
        ? Math.min(...sessionsWithPace.map((session) => session.pace_min_km || Number.POSITIVE_INFINITY))
        : null

    return {
      totalSessions: sessions.length,
      totalDistance,
      totalTime,
      averagePace,
      bestPace,
      longestRun,
    }
  }

  const loadStats = async () => {
    try {
      const sessions = await getRunningSessions()
      const calculatedStats = calculateStats(sessions || [])
      setStats(calculatedStats)
    } catch (error) {
      console.error("Error loading running stats:", error)
    } finally {
      setLoading(false)
    }
  }

  useEffect(() => {
    loadStats()
  }, [refreshTrigger])

  // Funcion para formatear el ritmo
  const formatPace = (pace: number | null) => {
    if (!pace) return "N/A"
    const minutes = Math.floor(pace)
    const seconds = Math.round((pace - minutes) * 60)
    return `${minutes}:${seconds.toString().padStart(2, "0")}`
  }

  // Funcion para formatear el tiempo
  const formatTime = (minutes: number) => {
    const hours = Math.floor(minutes / 60)
    const mins = minutes % 60
    if (hours > 0) {
      return `${hours}h ${mins}m`
    }
    return `${mins}m`
  }

  if (loading) {
    return (
      <Card>
        <CardContent className="p-6">
          <div className="text-center text-gray-500">Cargando estadisticas...</div>
        </CardContent>
      </Card>
    )
  }

  return (
    <Card>
      <CardHeader>
        <CardTitle>Estadisticas de Running</CardTitle>
        <CardDescription>Resumen de tu actividad de carrera</CardDescription>
      </CardHeader>
      <CardContent>
        <div className="grid grid-cols-2 md:grid-cols-3 gap-4">
          <div className="text-center">
            <div className="text-2xl font-bold text-green-600">{stats?.totalSessions}</div>
            <div className="text-sm text-gray-600">Sesiones Totales</div>
          </div>
          <div className="text-center">
            <div className="text-2xl font-bold text-green-600">{stats?.totalDistance.toFixed(1)} km</div>
            <div className="text-sm text-gray-600">Distancia Total</div>
          </div>
          <div className="text-center">
            <div className="text-2xl font-bold text-green-600">{formatTime(stats?.totalTime || 0)}</div>
            <div className="text-sm text-gray-600">Tiempo Total</div>
          </div>
          <div className="text-center">
            <div className="text-2xl font-bold text-green-600">{formatPace(stats?.averagePace)}</div>
            <div className="text-sm text-gray-600">Ritmo Promedio</div>
          </div>
          <div className="text-center">
            <div className="text-2xl font-bold text-green-600">{formatPace(stats?.bestPace)}</div>
            <div className="text-sm text-gray-600">Mejor Ritmo</div>
          </div>
          <div className="text-center">
            <div className="text-2xl font-bold text-green-600">{stats?.longestRun.toFixed(1)} km</div>
            <div className="text-sm text-gray-600">Carrera Mas Larga</div>
          </div>
        </div>
      </CardContent>
    </Card>
  )
}
\end{lstlisting}

\textbf{Metricas calculadas:}
\begin{itemize}
    \item \textbf{Sesiones Totales}: Numero total de sesiones registradas
    \item \textbf{Distancia Total}: Suma de todas las distancias recorridas
    \item \textbf{Tiempo Total}: Suma de todos los tiempos de carrera
    \item \textbf{Ritmo Promedio}: Promedio de todos los ritmos registrados
    \item \textbf{Mejor Ritmo}: El ritmo mas rapido registrado
    \item \textbf{Carrera Mas Larga}: La distancia mas larga en una sola sesion
\end{itemize}

\subsection{Componente RunningList}

Lista que muestra todas las sesiones de running registradas.

\begin{lstlisting}[caption=components/running/running-list.tsx - Estructura principal]
"use client"

import { useState, useEffect } from "react"
import { Card, CardContent } from "@/components/ui/card"
import { Button } from "@/components/ui/button"
import { Badge } from "@/components/ui/badge"
import { Trash2, Calendar, Clock, MapPin, Zap } from "lucide-react"
import { deleteRunningSession, getRunningSessions } from "@/lib/running-actions"
import { format } from "date-fns"
import { es } from "date-fns/locale"

interface RunningSession {
  id: string
  duration_minutes: number
  distance_km: number
  pace_min_km: number | null
  created_at: string
}

export default function RunningList({ refreshTrigger }: { refreshTrigger?: number }) {
  const [sessions, setSessions] = useState<RunningSession[]>([])
  const [loading, setLoading] = useState(true)

  const loadSessions = async () => {
    try {
      const data = await getRunningSessions()
      setSessions(data || [])
    } catch (error) {
      console.error("Error loading running sessions:", error)
    } finally {
      setLoading(false)
    }
  }

  useEffect(() => {
    loadSessions()
  }, [refreshTrigger])

  const handleDelete = async (id: string) => {
    if (confirm("Estas seguro de que quieres eliminar esta sesion?")) {
      const result = await deleteRunningSession(id)
      if (result?.success) {
        loadSessions()
      }
    }
  }

  const formatPace = (pace: number | null) => {
    if (!pace) return null
    const minutes = Math.floor(pace)
    const seconds = Math.round((pace - minutes) * 60)
    return `${minutes}:${seconds.toString().padStart(2, "0")}`
  }

  const formatTime = (minutes: number) => {
    const hours = Math.floor(minutes / 60)
    const mins = minutes % 60
    if (hours > 0) {
      return `${hours}h ${mins}m`
    }
    return `${mins}m`
  }

  // Estado de carga
  if (loading) {
    return (
      <Card>
        <CardContent className="p-6">
          <div className="text-center text-gray-500">Cargando sesiones...</div>
        </CardContent>
      </Card>
    )
  }

  // Estado vacio
  if (sessions.length === 0) {
    return (
      <Card>
        <CardContent className="p-6">
          <div className="text-center text-gray-500">
            <MapPin className="h-12 w-12 mx-auto mb-4 text-gray-300" />
            <p>No hay sesiones de running registradas</p>
            <p className="text-sm">Registra tu primera carrera para comenzar</p>
          </div>
        </CardContent>
      </Card>
    )
  }

  return (
    <div className="space-y-4">
      <h2 className="text-xl font-semibold">Historial de Sesiones</h2>
      {sessions.map((session) => (
        <Card key={session.id}>
          <CardContent className="p-4">
            <div className="flex items-center justify-between">
              <div className="flex-1">
                <div className="flex items-center gap-4 mb-2">
                  <div className="flex items-center gap-1 text-sm text-gray-600">
                    <Calendar className="h-4 w-4" />
                    {format(new Date(session.created_at), "dd/MM/yyyy", { locale: es })}
                  </div>
                  <div className="flex items-center gap-1 text-sm text-gray-600">
                    <Clock className="h-4 w-4" />
                    {formatTime(session.duration_minutes)}
                  </div>
                  <div className="flex items-center gap-1 text-sm text-gray-600">
                    <MapPin className="h-4 w-4" />
                    {session.distance_km.toFixed(1)} km
                  </div>
                  {session.pace_min_km && (
                    <div className="flex items-center gap-1 text-sm text-gray-600">
                      <Zap className="h-4 w-4" />
                      {formatPace(session.pace_min_km)}/km
                    </div>
                  )}
                </div>
                
                {/* Badges con metricas */}
                <div className="flex gap-2">
                  <Badge variant="secondary">
                    {session.distance_km.toFixed(1)} km
                  </Badge>
                  <Badge variant="secondary">
                    {formatTime(session.duration_minutes)}
                  </Badge>
                  {session.pace_min_km && (
                    <Badge variant="secondary">
                      {formatPace(session.pace_min_km)}/km
                    </Badge>
                  )}
                </div>
              </div>

              {/* Botones de accion */}
              <div className="flex gap-2">
                <Button
                  variant="outline"
                  size="sm"
                  onClick={() => handleDelete(session.id)}
                  className="text-red-600 hover:text-red-700"
                >
                  <Trash2 className="h-4 w-4" />
                </Button>
              </div>
            </div>
          </CardContent>
        </Card>
      ))}
    </div>
  )
}
\end{lstlisting}

\textbf{Caracteristicas de la lista:}
\begin{itemize}
    \item \textbf{Carga asincrona}: Carga las sesiones desde la base de datos
    \item \textbf{Formateo de datos}: Convierte tiempos y ritmos a formato legible
    \item \textbf{Estado vacio}: Muestra mensaje cuando no hay sesiones
    \item \textbf{Eliminacion}: Permite eliminar sesiones con confirmacion
    \item \textbf{Actualizacion automatica}: Se actualiza cuando se agregan nuevas sesiones
\end{itemize}

\subsection{Componente RunningCharts}

Componente para visualizar graficos de progreso y tendencias.

\begin{lstlisting}[caption=components/running/running-charts.tsx - Estructura basica]
"use client"

import { useState, useEffect } from "react"
import { Card, CardContent, CardHeader, CardTitle } from "@/components/ui/card"
import { getRunningSessions } from "@/lib/running-actions"

interface RunningSession {
  id: string
  duration_minutes: number
  distance_km: number
  pace_min_km: number | null
  created_at: string
}

export default function RunningCharts() {
  const [sessions, setSessions] = useState<RunningSession[]>([])
  const [loading, setLoading] = useState(true)

  useEffect(() => {
    const loadSessions = async () => {
      try {
        const data = await getRunningSessions()
        setSessions(data || [])
      } catch (error) {
        console.error("Error loading running sessions:", error)
      } finally {
        setLoading(false)
      }
    }

    loadSessions()
  }, [])

  if (loading) {
    return (
      <Card>
        <CardContent className="p-6">
          <div className="text-center text-gray-500">Cargando graficos...</div>
        </CardContent>
      </Card>
    )
  }

  if (sessions.length === 0) {
    return (
      <Card>
        <CardContent className="p-6">
          <div className="text-center text-gray-500">
            <p>No hay datos suficientes para mostrar graficos</p>
            <p className="text-sm">Registra algunas sesiones para ver tu progreso</p>
          </div>
        </CardContent>
      </Card>
    )
  }

  return (
    <div className="space-y-6">
      <Card>
        <CardHeader>
          <CardTitle>Progreso de Distancia</CardTitle>
        </CardHeader>
        <CardContent>
          {/* Aqui se implementarian los graficos con una libreria como Chart.js o Recharts */}
          <div className="h-64 flex items-center justify-center text-gray-500">
            Grafico de progreso de distancia (implementar con libreria de graficos)
          </div>
        </CardContent>
      </Card>

      <Card>
        <CardHeader>
          <CardTitle>Evolucion del Ritmo</CardTitle>
        </CardHeader>
        <CardContent>
          <div className="h-64 flex items-center justify-center text-gray-500">
            Grafico de evolucion del ritmo (implementar con libreria de graficos)
          </div>
        </CardContent>
      </Card>

      <Card>
        <CardHeader>
          <CardTitle>Frecuencia de Entrenamiento</CardTitle>
        </CardHeader>
        <CardContent>
          <div className="h-64 flex items-center justify-center text-gray-500">
            Grafico de frecuencia de entrenamiento (implementar con libreria de graficos)
          </div>
        </CardContent>
      </Card>
    </div>
  )
}
\end{lstlisting}

\textbf{Tipos de graficos implementables:}
\begin{itemize}
    \item \textbf{Progreso de Distancia}: Linea temporal mostrando la evolucion de las distancias
    \item \textbf{Evolucion del Ritmo}: Grafico de ritmo promedio por sesion
    \item \textbf{Frecuencia de Entrenamiento}: Histograma de sesiones por semana/mes
    \item \textbf{Comparacion de Metricas}: Graficos de barras comparando diferentes periodos
\end{itemize}

\section{Ejemplos Practicos de Uso}

\subsection{Ejemplos de Insercion a Base de Datos}

\subsubsection{Crear una Sesion de Running}

\begin{lstlisting}[style=sqlstyle, caption=Ejemplo de insercion en running_sessions]
-- Insertar una sesion de running de 5km en 30 minutos
INSERT INTO public.running_sessions (
    user_id,
    duration_minutes,
    distance_km,
    pace_min_km
) VALUES (
    '123e4567-e89b-12d3-a456-426614174000', -- UUID del usuario
    30,                                     -- 30 minutos
    5.0,                                    -- 5 kilometros
    6.0                                     -- 6 minutos por kilometro
);

-- Insertar una sesion de running de 10km en 50 minutos
INSERT INTO public.running_sessions (
    user_id,
    duration_minutes,
    distance_km,
    pace_min_km
) VALUES (
    '123e4567-e89b-12d3-a456-426614174000',
    50,                                     -- 50 minutos
    10.0,                                   -- 10 kilometros
    5.0                                     -- 5 minutos por kilometro
);

-- Insertar una sesion sin ritmo especificado (se calcula automaticamente)
INSERT INTO public.running_sessions (
    user_id,
    duration_minutes,
    distance_km
) VALUES (
    '123e4567-e89b-12d3-a456-426614174000',
    25,                                     -- 25 minutos
    4.2                                     -- 4.2 kilometros
    -- pace_min_km se calculara como 25/4.2 = 5.95 min/km
);
\end{lstlisting}

\subsubsection{Consultas Utiles para Analisis}

\begin{lstlisting}[style=sqlstyle, caption=Consultas utiles para analisis de running]
-- Obtener todas las sesiones de un usuario con detalles
SELECT 
    rs.duration_minutes,
    rs.distance_km,
    rs.pace_min_km,
    rs.created_at,
    (rs.distance_km / rs.duration_minutes * 60) as velocidad_kmh
FROM public.running_sessions rs
WHERE rs.user_id = '123e4567-e89b-12d3-a456-426614174000'
ORDER BY rs.created_at DESC;

-- Obtener estadisticas resumidas de un usuario
SELECT 
    COUNT(*) as total_sesiones,
    SUM(distance_km) as distancia_total,
    SUM(duration_minutes) as tiempo_total,
    AVG(pace_min_km) as ritmo_promedio,
    MIN(pace_min_km) as mejor_ritmo,
    MAX(distance_km) as carrera_mas_larga
FROM public.running_sessions
WHERE user_id = '123e4567-e89b-12d3-a456-426614174000';

-- Obtener progreso mensual
SELECT 
    DATE_TRUNC('month', created_at) as mes,
    COUNT(*) as sesiones_mes,
    SUM(distance_km) as distancia_mes,
    AVG(pace_min_km) as ritmo_promedio_mes
FROM public.running_sessions
WHERE user_id = '123e4567-e89b-12d3-a456-426614174000'
GROUP BY DATE_TRUNC('month', created_at)
ORDER BY mes DESC;

-- Obtener las 5 mejores sesiones por ritmo
SELECT 
    distance_km,
    duration_minutes,
    pace_min_km,
    created_at
FROM public.running_sessions
WHERE user_id = '123e4567-e89b-12d3-a456-426614174000'
  AND pace_min_km IS NOT NULL
ORDER BY pace_min_km ASC
LIMIT 5;

-- Obtener tendencia de mejora en el ultimo mes
SELECT 
    DATE_TRUNC('week', created_at) as semana,
    AVG(pace_min_km) as ritmo_promedio_semana
FROM public.running_sessions
WHERE user_id = '123e4567-e89b-12d3-a456-426614174000'
  AND created_at >= NOW() - INTERVAL '1 month'
  AND pace_min_km IS NOT NULL
GROUP BY DATE_TRUNC('week', created_at)
ORDER BY semana ASC;
\end{lstlisting}

\subsection{Flujos de Datos Completos}

\subsubsection{Flujo de Creacion de Sesion}

\begin{enumerate}
    \item \textbf{Usuario completa formulario} en \texttt{RunningForm}
    \item \textbf{Validacion en cliente} de campos requeridos (duracion y distancia)
    \item \textbf{Envio a Server Action} \texttt{createRunningSession}
    \item \textbf{Verificacion de autenticacion} en servidor
    \item \textbf{Calculo automatico de ritmo} si no se proporciona
    \item \textbf{Insercion en base de datos} tabla \texttt{running\_sessions}
    \item \textbf{Revalidacion de cache} con \texttt{revalidatePath}
    \item \textbf{Actualizacion de UI} con nueva sesion y estadisticas
\end{enumerate}

\subsubsection{Flujo de Calculo de Estadisticas}

\begin{enumerate}
    \item \textbf{Carga de sesiones} con \texttt{getRunningSessions}
    \item \textbf{Calculo de metricas basicas} (total sesiones, distancia, tiempo)
    \item \textbf{Calculo de ritmo promedio} de sesiones con ritmo registrado
    \item \textbf{Identificacion del mejor ritmo} (valor minimo)
    \item \textbf{Identificacion de la carrera mas larga} (distancia maxima)
    \item \textbf{Formateo de datos} para presentacion
    \item \textbf{Renderizado de estadisticas} en \texttt{RunningStats}
\end{enumerate}

\section{Caracteristicas Avanzadas}

\subsection{Calculos Automaticos}

El modulo implementa varios calculos automaticos:

\begin{itemize}
    \item \textbf{Ritmo por Kilometro}: \texttt{duracion\_minutos / distancia\_km}
    \item \textbf{Velocidad en km/h}: \texttt{distancia\_km / (duracion\_minutos / 60)}
    \item \textbf{Ritmo Promedio}: Promedio de todos los ritmos registrados
    \item \textbf{Mejor Ritmo}: Valor minimo de ritmo (mas rapido)
    \item \textbf{Distancia Total}: Suma de todas las distancias
    \item \textbf{Tiempo Total}: Suma de todos los tiempos
\end{itemize}

\subsection{Validaciones y Restricciones}

\begin{itemize}
    \item \textbf{Duracion}: Debe ser mayor a 0 minutos
    \item \textbf{Distancia}: Debe ser mayor a 0 kilometros
    \item \textbf{Ritmo}: Opcional, se calcula automaticamente si no se proporciona
    \item \textbf{Usuario}: Debe estar autenticado para todas las operaciones
    \item \textbf{RLS}: Solo se puede acceder a sesiones propias
\end{itemize}

\subsection{Formateo de Datos}

\begin{itemize}
    \item \textbf{Tiempo}: Convierte minutos a formato "Xh Ym" o "Ym"
    \item \textbf{Ritmo}: Convierte decimal a formato "X:YY" (minutos:segundos)
    \item \textbf{Distancia}: Muestra con 1 decimal (ej: "5.0 km")
    \item \textbf{Fechas}: Formato localizado (ej: "14/10/2025")
\end{itemize}

\section{Mejores Practicas de Desarrollo}

\subsection{Seguridad}

\begin{itemize}
    \item \textbf{Validacion de entrada}: Siempre validar duracion y distancia
    \item \textbf{Autenticacion}: Verificar usuario en cada operacion
    \item \textbf{RLS}: Usar Row Level Security en todas las tablas
    \item \textbf{Sanitizacion}: Limpiar y convertir datos de entrada
    \item \textbf{Logs de seguridad}: Registrar operaciones sensibles
\end{itemize}

\subsection{Performance}

\begin{itemize}
    \item \textbf{Indices de base de datos}: Optimizar consultas por usuario y fecha
    \item \textbf{Cache}: Usar revalidatePath para actualizar cache
    \item \textbf{Calculos eficientes}: Optimizar calculos de estadisticas
    \item \textbf{Lazy loading}: Cargar graficos bajo demanda
    \item \textbf{Paginacion}: Implementar para listas largas de sesiones
\end{itemize}

\subsection{Mantenibilidad}

\begin{itemize}
    \item \textbf{TypeScript}: Usar tipado estricto en todos los componentes
    \item \textbf{Interfaces}: Definir interfaces claras para props
    \item \textbf{Separacion de responsabilidades}: Logica en Server Actions
    \item \textbf{Reutilizacion}: Componentes modulares y reutilizables
    \item \textbf{Documentacion}: Comentar codigo complejo
\end{itemize}

\section{Solucion de Problemas}

\subsection{Problemas Comunes}

\subsubsection{Error: Tabla no existe}

\textbf{Sintomas}: Error al intentar insertar o consultar datos
\textbf{Causa}: Las tablas no han sido creadas en la base de datos
\textbf{Solucion}: Ejecutar los scripts SQL en orden:
\begin{enumerate}
    \item \texttt{01-create-database-schema.sql}
    \item \texttt{01-create-user-schema.sql}
    \item \texttt{02-create-history-schema.sql}
\end{enumerate}

\subsubsection{Error: Usuario no autenticado}

\textbf{Sintomas}: Server Actions retornan error de autenticacion
\textbf{Causa}: Usuario no esta logueado o sesion expirada
\textbf{Solucion}: Verificar estado de autenticacion y redirigir a login

\subsubsection{Error: Calculo de ritmo incorrecto}

\textbf{Sintomas}: El ritmo calculado no coincide con el esperado
\textbf{Causa}: Division por cero o valores invalidos
\textbf{Solucion}: Validar que distancia sea mayor a 0 antes del calculo

\subsection{Debugging}

\subsubsection{Logs del Cliente}

\begin{lstlisting}[caption=Debugging en cliente]
// Habilitar logs detallados
localStorage.setItem('debug', 'true');

// Verificar estado de autenticacion
console.log('User:', user);
console.log('Session:', session);

// Verificar datos de sesiones
console.log('Running Sessions:', sessions);
\end{lstlisting}

\subsubsection{Logs del Servidor}

\begin{lstlisting}[caption=Debugging en servidor]
// En Server Actions
console.log('Action called with:', { userId, data });

// En consultas de base de datos
console.log('Query result:', { data, error });

// En calculos de ritmo
console.log('Pace calculation:', { duration, distance, calculatedPace });
\end{lstlisting}

\section{Conclusion}

El modulo de running de FitTrack es un sistema completo y robusto que permite a los usuarios gestionar sus sesiones de carrera de manera eficiente. Con su arquitectura bien definida, base de datos optimizada y componentes React modernos, proporciona una experiencia de usuario excepcional.

\textbf{Caracteristicas destacadas:}
\begin{itemize}
    \item Arquitectura escalable y mantenible
    \item Seguridad robusta con RLS
    \item Interfaz de usuario intuitiva
    \item Calculos automaticos precisos
    \item Estadisticas detalladas y utiles
    \item Codigo bien documentado y tipado
\end{itemize}

Para contribuir al desarrollo del modulo:
\begin{enumerate}
    \item Seguir las convenciones establecidas
    \item Implementar tests apropiados
    \item Documentar cambios significativos
    \item Mantener la compatibilidad con la base de datos
    \item Respetar las politicas de seguridad
\end{enumerate}

\end{document}
