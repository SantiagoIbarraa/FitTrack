\documentclass[12pt,a4paper]{article}
\usepackage[utf8]{inputenc}
\usepackage[spanish]{babel}
\usepackage{geometry}
\usepackage{graphicx}
\usepackage{hyperref}
\usepackage{listings}
\usepackage{xcolor}
\usepackage{booktabs}
\usepackage{array}
\usepackage{longtable}
\usepackage{fancyhdr}
\usepackage{titlesec}
\usepackage{enumitem}
\usepackage{amsmath}
\usepackage{amsfonts}
\usepackage{amssymb}
\usepackage{float}

% Configuración de página
\geometry{margin=2.5cm}
\pagestyle{fancy}
\fancyhf{}
\fancyhead[L]{Módulo de Gimnasio - FitTrack}
\fancyhead[R]{\thepage}
\renewcommand{\headrulewidth}{0.4pt}

% Configuración de colores para código
\definecolor{codegreen}{rgb}{0,0.6,0}
\definecolor{codegray}{rgb}{0.5,0.5,0.5}
\definecolor{codepurple}{rgb}{0.58,0,0.82}
\definecolor{backcolour}{rgb}{0.95,0.95,0.92}
\definecolor{bluekeywords}{rgb}{0.13,0.13,1}
\definecolor{greencomments}{rgb}{0,0.5,0}
\definecolor{redstrings}{rgb}{0.9,0,0}

% Configuración de listings
\lstdefinestyle{mystyle}{
    backgroundcolor=\color{backcolour},   
    commentstyle=\color{greencomments},
    keywordstyle=\color{bluekeywords},
    numberstyle=\tiny\color{codegray},
    stringstyle=\color{redstrings},
    basicstyle=\ttfamily\footnotesize,
    breakatwhitespace=false,         
    breaklines=true,                 
    captionpos=b,                    
    keepspaces=true,                 
    numbers=left,                    
    numbersep=5pt,                  
    showspaces=false,                
    showstringspaces=false,
    showtabs=false,                  
    tabsize=2,
    frame=single,
    rulecolor=\color{codegray}
}

\lstdefinestyle{sqlstyle}{
    backgroundcolor=\color{backcolour},   
    commentstyle=\color{greencomments},
    keywordstyle=\color{bluekeywords},
    numberstyle=\tiny\color{codegray},
    stringstyle=\color{redstrings},
    basicstyle=\ttfamily\footnotesize,
    breakatwhitespace=false,         
    breaklines=true,                 
    captionpos=b,                    
    keepspaces=true,                 
    numbers=left,                    
    numbersep=5pt,                  
    showspaces=false,                
    showstringspaces=false,
    showtabs=false,                  
    tabsize=2,
    frame=single,
    rulecolor=\color{codegray},
    language=SQL
}

\lstset{style=mystyle}

% Configuración de títulos
\titleformat{\section}{\Large\bfseries}{\thesection}{1em}{}
\titleformat{\subsection}{\large\bfseries}{\thesubsection}{1em}{}
\titleformat{\subsubsection}{\normalsize\bfseries}{\thesubsubsection}{1em}{}

% Configuración de hipervínculos
\hypersetup{
    colorlinks=true,
    linkcolor=blue,
    filecolor=magenta,      
    urlcolor=cyan,
    citecolor=red,
}

\begin{document}

% Portada
\begin{titlepage}
\centering
\vspace*{2cm}

{\Huge\bfseries Módulo de Gimnasio}\\[0.5cm]
{\LARGE FitTrack}\\[1cm]

{\large Documentación Técnica Completa}\\[2cm]

\begin{minipage}{0.8\textwidth}
\centering
Este documento proporciona una guía técnica exhaustiva del módulo de gimnasio de FitTrack, incluyendo arquitectura, implementación, base de datos, componentes React, Server Actions y ejemplos prácticos de uso.
\end{minipage}

\vfill

{\large Versión 1.0}\\[0.5cm]
{\large \today}

\end{titlepage}

\tableofcontents
\newpage

\section{Introducción al Módulo de Gimnasio}

El módulo de gimnasio es uno de los componentes más complejos y funcionales de FitTrack. Permite a los usuarios registrar entrenamientos individuales, crear y gestionar rutinas personalizadas, hacer seguimiento de su progreso y acceder a un catálogo de ejercicios administrado.

\subsection{Características Principales}

\begin{itemize}
    \item \textbf{Registro de Entrenamientos}: Permite registrar ejercicios individuales con peso, repeticiones y series
    \item \textbf{Gestión de Rutinas}: Crear, editar y ejecutar rutinas personalizadas
    \item \textbf{Catálogo de Ejercicios}: Base de datos de ejercicios administrada por administradores
    \item \textbf{Historial de Progreso}: Seguimiento detallado del progreso a lo largo del tiempo
    \item \textbf{Métricas y Estadísticas}: Análisis de rendimiento y tendencias
    \item \textbf{Selector de Ejercicios}: Interfaz intuitiva para seleccionar ejercicios del catálogo
    \item \textbf{Imágenes de Ejercicios}: Soporte para imágenes en ejercicios personalizados
\end{itemize}

\subsection{Arquitectura General}

El módulo sigue una arquitectura de capas bien definida:

\begin{enumerate}
    \item \textbf{Capa de Presentación}: Componentes React con TypeScript
    \item \textbf{Capa de Lógica}: Server Actions de Next.js
    \item \textbf{Capa de Datos}: Supabase PostgreSQL con RLS
    \item \textbf{Capa de Servicios}: Utilidades y helpers
\end{enumerate}

\section{Estructura de Base de Datos}

\subsection{Diagrama de Relaciones}

\begin{figure}[H]
\centering
\begin{verbatim}
    auth.users
         |
    +----+----+
    |         |
gym_workouts routines
    |         |
    |    routine_exercises
    |
gym_exercises
\end{verbatim}
\caption{Diagrama de relaciones de las tablas del módulo de gimnasio}
\end{figure}

\subsection{Tabla gym\_workouts}

Esta tabla almacena los entrenamientos individuales realizados por los usuarios.

\begin{lstlisting}[style=sqlstyle, caption=Estructura completa de gym_workouts]
CREATE TABLE IF NOT EXISTS public.gym_workouts (
    id UUID PRIMARY KEY DEFAULT uuid_generate_v4(),
    user_id UUID REFERENCES auth.users(id) ON DELETE CASCADE NOT NULL,
    exercise_name TEXT NOT NULL,
    weight_kg NUMERIC(5,2),
    repetitions INTEGER,
    sets INTEGER,
    image_url TEXT,
    created_at TIMESTAMP WITH TIME ZONE DEFAULT NOW()
);

-- Índices para optimización
CREATE INDEX IF NOT EXISTS idx_gym_workouts_user_id ON public.gym_workouts(user_id);
CREATE INDEX IF NOT EXISTS idx_gym_workouts_created_at ON public.gym_workouts(created_at);
CREATE INDEX IF NOT EXISTS idx_gym_workouts_exercise_name ON public.gym_workouts(exercise_name);
\end{lstlisting}

\textbf{Descripción de campos:}
\begin{itemize}
    \item \texttt{id}: Identificador único UUID generado automáticamente
    \item \texttt{user\_id}: Referencia al usuario propietario del entrenamiento
    \item \texttt{exercise\_name}: Nombre del ejercicio realizado (texto libre)
    \item \texttt{weight\_kg}: Peso utilizado en kilogramos (opcional, máximo 999.99)
    \item \texttt{repetitions}: Número de repeticiones realizadas (opcional)
    \item \texttt{sets}: Número de series realizadas (opcional)
    \item \texttt{image\_url}: URL de imagen del ejercicio (opcional)
    \item \texttt{created\_at}: Timestamp de creación automático
\end{itemize}

\subsection{Tabla routines}

Almacena las rutinas personalizadas creadas por los usuarios.

\begin{lstlisting}[style=sqlstyle, caption=Estructura completa de routines]
CREATE TABLE IF NOT EXISTS public.routines (
    id UUID DEFAULT gen_random_uuid() PRIMARY KEY,
    user_id UUID REFERENCES auth.users(id) ON DELETE CASCADE NOT NULL,
    name TEXT NOT NULL,
    description TEXT,
    created_at TIMESTAMP WITH TIME ZONE DEFAULT TIMEZONE('utc'::text, NOW()) NOT NULL,
    updated_at TIMESTAMP WITH TIME ZONE DEFAULT TIMEZONE('utc'::text, NOW()) NOT NULL
);

-- Índices para optimización
CREATE INDEX IF NOT EXISTS idx_routines_user_id ON public.routines(user_id);
CREATE INDEX IF NOT EXISTS idx_routines_created_at ON public.routines(created_at);
\end{lstlisting}

\textbf{Descripción de campos:}
\begin{itemize}
    \item \texttt{id}: Identificador único UUID de la rutina
    \item \texttt{user\_id}: Referencia al usuario propietario
    \item \texttt{name}: Nombre de la rutina (requerido)
    \item \texttt{description}: Descripción opcional de la rutina
    \item \texttt{created\_at}: Timestamp de creación
    \item \texttt{updated\_at}: Timestamp de última modificación
\end{itemize}

\subsection{Tabla routine\_exercises}

Contiene los ejercicios que pertenecen a cada rutina con su orden específico.

\begin{lstlisting}[style=sqlstyle, caption=Estructura completa de routine_exercises]
CREATE TABLE IF NOT EXISTS public.routine_exercises (
    id UUID DEFAULT gen_random_uuid() PRIMARY KEY,
    routine_id UUID REFERENCES public.routines(id) ON DELETE CASCADE NOT NULL,
    exercise_name TEXT NOT NULL,
    weight DECIMAL(6,2) NOT NULL CHECK (weight >= 0),
    repetitions INTEGER NOT NULL CHECK (repetitions > 0),
    sets INTEGER NOT NULL CHECK (sets > 0),
    image_url TEXT,
    order_index INTEGER NOT NULL DEFAULT 0,
    created_at TIMESTAMP WITH TIME ZONE DEFAULT TIMEZONE('utc'::text, NOW()) NOT NULL
);

-- Índices para optimización
CREATE INDEX IF NOT EXISTS idx_routine_exercises_routine_id ON public.routine_exercises(routine_id);
CREATE INDEX IF NOT EXISTS idx_routine_exercises_order ON public.routine_exercises(routine_id, order_index);
\end{lstlisting}

\textbf{Descripción de campos:}
\begin{itemize}
    \item \texttt{id}: Identificador único del ejercicio en la rutina
    \item \texttt{routine\_id}: Referencia a la rutina padre
    \item \texttt{exercise\_name}: Nombre del ejercicio
    \item \texttt{weight}: Peso en kilogramos (requerido, >= 0)
    \item \texttt{repetitions}: Repeticiones (requerido, > 0)
    \item \texttt{sets}: Series (requerido, > 0)
    \item \texttt{image\_url}: URL de imagen opcional
    \item \texttt{order\_index}: Orden del ejercicio en la rutina
    \item \texttt{created\_at}: Timestamp de creación
\end{itemize}

\subsection{Tabla gym\_exercises}

Catálogo de ejercicios administrado por administradores del sistema.

\begin{lstlisting}[style=sqlstyle, caption=Estructura completa de gym_exercises]
CREATE TABLE IF NOT EXISTS gym_exercises (
    id UUID PRIMARY KEY DEFAULT gen_random_uuid(),
    name TEXT NOT NULL,
    category TEXT NOT NULL CHECK (category IN (
        'Pecho', 'Bíceps', 'Tríceps', 'Hombros', 
        'Pierna', 'Espalda', 'Otros'
    )),
    description TEXT,
    image_url TEXT,
    created_at TIMESTAMP WITH TIME ZONE DEFAULT NOW(),
    updated_at TIMESTAMP WITH TIME ZONE DEFAULT NOW()
);

-- Índices para optimización
CREATE INDEX IF NOT EXISTS idx_gym_exercises_category ON gym_exercises(category);
CREATE INDEX IF NOT EXISTS idx_gym_exercises_name ON gym_exercises(name);
\end{lstlisting}

\textbf{Descripción de campos:}
\begin{itemize}
    \item \texttt{id}: Identificador único del ejercicio
    \item \texttt{name}: Nombre del ejercicio
    \item \texttt{category}: Categoría del ejercicio (constraint de valores válidos)
    \item \texttt{description}: Descripción detallada del ejercicio
    \item \texttt{image\_url}: URL de imagen del ejercicio
    \item \texttt{created\_at}: Timestamp de creación
    \item \texttt{updated\_at}: Timestamp de última modificación
\end{itemize}

\subsection{Políticas de Seguridad (RLS)}

Todas las tablas implementan Row Level Security para garantizar que los usuarios solo accedan a sus propios datos.

\begin{lstlisting}[style=sqlstyle, caption=Políticas RLS para gym_workouts]
-- Habilitar RLS
ALTER TABLE public.gym_workouts ENABLE ROW LEVEL SECURITY;

-- Políticas para gym_workouts
CREATE POLICY "Usuarios pueden ver sus propios entrenamientos"
ON public.gym_workouts
FOR SELECT USING (auth.uid() = user_id);

CREATE POLICY "Usuarios pueden insertar sus propios entrenamientos"
ON public.gym_workouts
FOR INSERT WITH CHECK (auth.uid() = user_id);

CREATE POLICY "Usuarios pueden modificar sus propios entrenamientos"
ON public.gym_workouts
FOR UPDATE USING (auth.uid() = user_id);

CREATE POLICY "Usuarios pueden eliminar sus propios entrenamientos"
ON public.gym_workouts
FOR DELETE USING (auth.uid() = user_id);
\end{lstlisting}

\begin{lstlisting}[style=sqlstyle, caption=Políticas RLS para routines]
-- Habilitar RLS
ALTER TABLE public.routines ENABLE ROW LEVEL SECURITY;

-- Políticas para routines
CREATE POLICY "Users can view own routines" ON public.routines
FOR SELECT USING (auth.uid() = user_id);

CREATE POLICY "Users can insert own routines" ON public.routines
FOR INSERT WITH CHECK (auth.uid() = user_id);

CREATE POLICY "Users can update own routines" ON public.routines
FOR UPDATE USING (auth.uid() = user_id);

CREATE POLICY "Users can delete own routines" ON public.routines
FOR DELETE USING (auth.uid() = user_id);
\end{lstlisting}

\begin{lstlisting}[style=sqlstyle, caption=Políticas RLS para routine_exercises]
-- Habilitar RLS
ALTER TABLE public.routine_exercises ENABLE ROW LEVEL SECURITY;

-- Políticas para routine_exercises
CREATE POLICY "Users can view own routine exercises" ON public.routine_exercises
FOR SELECT USING (
    EXISTS (
        SELECT 1 FROM public.routines 
        WHERE routines.id = routine_exercises.routine_id 
        AND routines.user_id = auth.uid()
    )
);

CREATE POLICY "Users can insert own routine exercises" ON public.routine_exercises
FOR INSERT WITH CHECK (
    EXISTS (
        SELECT 1 FROM public.routines 
        WHERE routines.id = routine_exercises.routine_id 
        AND routines.user_id = auth.uid()
    )
);

CREATE POLICY "Users can update own routine exercises" ON public.routine_exercises
FOR UPDATE USING (
    EXISTS (
        SELECT 1 FROM public.routines 
        WHERE routines.id = routine_exercises.routine_id 
        AND routines.user_id = auth.uid()
    )
);

CREATE POLICY "Users can delete own routine exercises" ON public.routine_exercises
FOR DELETE USING (
    EXISTS (
        SELECT 1 FROM public.routines 
        WHERE routines.id = routine_exercises.routine_id 
        AND routines.user_id = auth.uid()
    )
);
\end{lstlisting}

\begin{lstlisting}[style=sqlstyle, caption=Políticas RLS para gym_exercises]
-- Habilitar RLS
ALTER TABLE gym_exercises ENABLE ROW LEVEL SECURITY;

-- Permitir a todos los usuarios autenticados ver ejercicios
CREATE POLICY "Anyone can view gym exercises"
ON gym_exercises
FOR SELECT
TO authenticated
USING (true);

-- Solo administradores pueden gestionar ejercicios
CREATE POLICY "Only admins can manage gym exercises"
ON gym_exercises
FOR ALL
TO authenticated
USING (
    EXISTS (
        SELECT 1 FROM user_profiles
        WHERE user_profiles.id = auth.uid()
        AND user_profiles.role = 'admin'
    )
);
\end{lstlisting}

\section{Server Actions - Lógica de Negocio}

Las Server Actions manejan toda la lógica de negocio del módulo de gimnasio. Están implementadas en TypeScript con validación robusta y manejo de errores.

\subsection{Archivo gym-actions.ts}

\subsubsection{createWorkout - Crear Entrenamiento}

\begin{lstlisting}[language=typescript, caption=Función createWorkout completa]
"use server"

import { revalidatePath } from "next/cache"
import { createClient } from "@/lib/supabase/server"

export async function createWorkout(prevState: any, formData: FormData) {
  // Extraer datos del formulario
  const exercise_name = formData.get("exercise_name")?.toString()
  const weight_kg = formData.get("weight_kg")?.toString()
  const repetitions = formData.get("repetitions")?.toString()
  const sets = formData.get("sets")?.toString()
  const image_url = formData.get("image_url")?.toString()

  // Validación básica
  if (!exercise_name) {
    return { error: "El nombre del ejercicio es requerido" }
  }

  // Verificar autenticación
  const supabase = await createClient()
  const {
    data: { user },
  } = await supabase.auth.getUser()

  if (!user) {
    return { error: "Usuario no autenticado" }
  }

  try {
    // Preparar datos para inserción
    const insertData = {
      user_id: user.id,
      exercise_name,
      weight_kg: weight_kg && weight_kg.trim() !== "" ? 
        Math.max(0, Number.parseFloat(weight_kg)) : null,
      repetitions: repetitions && repetitions.trim() !== "" ? 
        Math.max(1, Number.parseInt(repetitions)) : null,
      sets: sets && sets.trim() !== "" ? 
        Math.max(1, Number.parseInt(sets)) : null,
      image_url: image_url && image_url.trim() !== "" ? 
        image_url.trim() : null,
    }

    // Insertar en base de datos
    const { error } = await supabase
      .from("gym_workouts")
      .insert(insertData)

    if (error) {
      console.error("Database error:", error)
      return { error: "Error al guardar el ejercicio" }
    }

    // Revalidar cache
    revalidatePath("/gym")
    return { success: true }
  } catch (error) {
    console.error("Error:", error)
    return { error: "Error al guardar el ejercicio" }
  }
}
\end{lstlisting}

\textbf{Características importantes:}
\begin{itemize}
    \item \textbf{Validación de entrada}: Verifica que el nombre del ejercicio esté presente
    \item \textbf{Autenticación}: Confirma que el usuario esté autenticado
    \item \textbf{Sanitización}: Convierte y valida valores numéricos
    \item \textbf{Manejo de errores}: Captura y reporta errores de base de datos
    \item \textbf{Revalidación}: Actualiza el cache de Next.js
\end{itemize}

\subsubsection{updateWorkout - Actualizar Entrenamiento}

\begin{lstlisting}[language=typescript, caption=Función updateWorkout completa]
export async function updateWorkout(prevState: any, formData: FormData) {
  const id = formData.get("id")?.toString()
  const exercise_name = formData.get("exercise_name")?.toString()
  const weight_kg = formData.get("weight_kg")?.toString()
  const repetitions = formData.get("repetitions")?.toString()
  const sets = formData.get("sets")?.toString()
  const image_url = formData.get("image_url")?.toString()

  // Validación de ID y nombre
  if (!id || !exercise_name) {
    return { error: "ID y nombre del ejercicio son requeridos" }
  }

  const supabase = await createClient()
  const {
    data: { user },
  } = await supabase.auth.getUser()

  if (!user) {
    return { error: "Usuario no autenticado" }
  }

  try {
    // Actualizar registro
    const { error } = await supabase
      .from("gym_workouts")
      .update({
        exercise_name,
        weight_kg: weight_kg && weight_kg.trim() !== "" ? 
          Math.max(0, Number.parseFloat(weight_kg)) : null,
        repetitions: repetitions && repetitions.trim() !== "" ? 
          Math.max(1, Number.parseInt(repetitions)) : null,
        sets: sets && sets.trim() !== "" ? 
          Math.max(1, Number.parseInt(sets)) : null,
        image_url: image_url && image_url.trim() !== "" ? 
          image_url.trim() : null,
      })
      .eq("id", id)
      .eq("user_id", user.id) // Seguridad adicional

    if (error) {
      console.error("Database error:", error)
      return { error: "Error al actualizar el ejercicio" }
    }

    revalidatePath("/gym")
    return { success: true }
  } catch (error) {
    console.error("Error:", error)
    return { error: "Error al actualizar el ejercicio" }
  }
}
\end{lstlisting}

\subsubsection{getWorkouts - Obtener Entrenamientos}

\begin{lstlisting}[language=typescript, caption=Función getWorkouts completa]
export async function getWorkouts() {
  const supabase = await createClient()
  const {
    data: { user },
  } = await supabase.auth.getUser()

  if (!user) {
    return []
  }

  try {
    const { data, error } = await supabase
      .from("gym_workouts")
      .select("*")
      .eq("user_id", user.id)
      .order("created_at", { ascending: false })

    if (error) {
      console.error("Database error:", error)
      return []
    }

    return data || []
  } catch (error) {
    console.error("Error:", error)
    return []
  }
}
\end{lstlisting}

\subsubsection{deleteWorkout - Eliminar Entrenamiento}

\begin{lstlisting}[language=typescript, caption=Función deleteWorkout completa]
export async function deleteWorkout(workoutId: string) {
  const supabase = await createClient()
  const {
    data: { user },
  } = await supabase.auth.getUser()

  if (!user) {
    return { error: "Usuario no autenticado" }
  }

  try {
    const { error } = await supabase
      .from("gym_workouts")
      .delete()
      .eq("id", workoutId)
      .eq("user_id", user.id) // Seguridad adicional

    if (error) {
      console.error("Database error:", error)
      return { error: "Error al eliminar el ejercicio" }
    }

    revalidatePath("/gym")
    return { success: true }
  } catch (error) {
    console.error("Error:", error)
    return { error: "Error al eliminar el ejercicio" }
  }
}
\end{lstlisting}

\subsection{Archivo routine-actions.ts}

\subsubsection{createRoutine - Crear Rutina}

\begin{lstlisting}[language=typescript, caption=Función createRoutine completa]
export async function createRoutine(prevState: any, formData: FormData) {
  const name = formData.get("name")?.toString()
  const description = formData.get("description")?.toString() || ""

  if (!name) {
    return { error: "El nombre de la rutina es requerido" }
  }

  try {
    const supabase = await createClient()
    const {
      data: { user },
      error: userError,
    } = await supabase.auth.getUser()

    if (userError) {
      console.error("User auth error:", userError)
      return { error: `Error de autenticación: ${userError.message}` }
    }

    if (!user) {
      return { error: "Usuario no autenticado" }
    }

    // Verificar que la tabla routines existe
    const { data: tableCheck, error: tableError } = await supabase
      .from("routines")
      .select("id")
      .limit(1)

    if (tableError) {
      console.error("Table check error:", tableError)
      return { error: `Error al verificar tabla: ${tableError.message}` }
    }

    // Insertar nueva rutina
    const { data: insertData, error: insertError } = await supabase
      .from("routines")
      .insert({
        user_id: user.id,
        name,
        description,
      })
      .select()

    if (insertError) {
      console.error("Database insert error:", insertError)
      return { error: `Error al crear la rutina: ${insertError.message}` }
    }

    revalidatePath("/gym")
    return { success: true, data: insertData }
  } catch (error) {
    console.error("Unexpected error in createRoutine:", error)
    return { error: `Error inesperado: ${error instanceof Error ? error.message : String(error)}` }
  }
}
\end{lstlisting}

\subsubsection{getRoutines - Obtener Rutinas}

\begin{lstlisting}[language=typescript, caption=Función getRoutines completa]
export async function getRoutines() {
  const supabase = await createClient()
  const {
    data: { user },
  } = await supabase.auth.getUser()

  if (!user) {
    return []
  }

  try {
    const { data, error } = await supabase
      .from("routines")
      .select(`
        *,
        routine_exercises(count)
      `)
      .eq("user_id", user.id)
      .order("created_at", { ascending: false })

    if (error) {
      console.error("Database error:", error)
      return []
    }

    return data || []
  } catch (error) {
    console.error("Error:", error)
    return []
  }
}
\end{lstlisting}

\subsubsection{addExerciseToRoutine - Agregar Ejercicio a Rutina}

\begin{lstlisting}[language=typescript, caption=Función addExerciseToRoutine completa]
export async function addExerciseToRoutine(routineId: string, exerciseData: any) {
  const supabase = await createClient()
  const {
    data: { user },
  } = await supabase.auth.getUser()

  if (!user) {
    return { error: "Usuario no autenticado" }
  }

  try {
    // Obtener el siguiente order_index para esta rutina
    const { data: maxOrderData } = await supabase
      .from("routine_exercises")
      .select("order_index")
      .eq("routine_id", routineId)
      .order("order_index", { ascending: false })
      .limit(1)

    const nextOrderIndex = maxOrderData && maxOrderData.length > 0 ? 
      maxOrderData[0].order_index + 1 : 0

    // Validar y preparar datos
    const sets = Math.max(1, Number.parseInt(exerciseData.sets) || 1)
    const repetitions = Math.max(1, Number.parseInt(exerciseData.repetitions) || 1)
    const weight = Math.max(0, Number.parseFloat(exerciseData.weight) || 0)

    // Insertar ejercicio en rutina
    const { error } = await supabase
      .from("routine_exercises")
      .insert({
        routine_id: routineId,
        exercise_name: exerciseData.exercise_name,
        weight: weight,
        repetitions: repetitions,
        sets: sets,
        image_url: exerciseData.image_url || null,
        order_index: nextOrderIndex,
      })

    if (error) {
      console.error("Database error:", error)
      return { error: "Error al agregar ejercicio a la rutina" }
    }

    revalidatePath("/gym")
    return { success: true }
  } catch (error) {
    console.error("Error:", error)
    return { error: "Error al agregar ejercicio a la rutina" }
  }
}
\end{lstlisting}

\section{Componentes React}

\subsection{Página Principal - GymPage}

El componente principal que coordina toda la funcionalidad del módulo de gimnasio.

\begin{lstlisting}[language=typescript, caption=app/gym/page.tsx - Estructura completa]
"use client"

import { useState } from "react"
import { Dumbbell, ArrowLeft } from "lucide-react"
import Link from "next/link"
import { Button } from "@/components/ui/button"
import WorkoutForm from "@/components/gym/workout-form"
import WorkoutList from "@/components/gym/workout-list"
import RoutineList from "@/components/gym/routine-list"
import RoutineForm from "@/components/gym/routine-form"
import RoutineDetail from "@/components/gym/routine-detail"
import ExerciseHistory from "@/components/gym/exercise-history"
import GymMetrics from "@/components/gym/gym-metrics"

// Interfaces TypeScript
interface Workout {
  id: string
  exercise_name: string
  weight_kg: number | null
  repetitions: number | null
  sets: number | null
  created_at: string
}

type ViewMode = "routines" | "individual" | "routine-detail" | "create-routine" | "history" | "metrics"

export default function GymPage() {
  // Estados del componente
  const [refreshTrigger, setRefreshTrigger] = useState(0)
  const [editingWorkout, setEditingWorkout] = useState<Workout | null>(null)
  const [viewMode, setViewMode] = useState<ViewMode>("routines")
  const [selectedRoutine, setSelectedRoutine] = useState<{ id: string; name: string } | null>(null)

  // Handlers para diferentes acciones
  const handleWorkoutAdded = () => {
    setRefreshTrigger((prev) => prev + 1)
  }

  const handleEditWorkout = (workout: Workout) => {
    setEditingWorkout(workout)
  }

  const handleEditComplete = () => {
    setEditingWorkout(null)
    setRefreshTrigger((prev) => prev + 1)
  }

  const handleViewRoutine = (routineId: string, routineName: string) => {
    setSelectedRoutine({ id: routineId, name: routineName })
    setViewMode("routine-detail")
  }

  const handleCreateRoutine = () => {
    setViewMode("create-routine")
  }

  const handleRoutineCreated = () => {
    setViewMode("routines")
    setRefreshTrigger((prev) => prev + 1)
  }

  const handleBackToRoutines = () => {
    setSelectedRoutine(null)
    setViewMode("routines")
    setRefreshTrigger((prev) => prev + 1)
  }

  return (
    <div className="min-h-screen bg-gradient-to-br from-blue-50 to-indigo-100 dark:from-gray-900 dark:to-gray-800">
      <div className="container mx-auto px-4 py-8 max-w-4xl">
        {/* Header con navegación */}
        <div className="mb-8">
          <div className="flex items-center gap-4 mb-4">
            <Button variant="outline" size="sm" asChild>
              <Link href="/">
                <ArrowLeft className="h-4 w-4 mr-2" />
                Volver
              </Link>
            </Button>
          </div>
          <div className="flex items-center gap-3 mb-2">
            <Dumbbell className="h-8 w-8 text-blue-600 dark:text-blue-400" />
            <h1 className="text-3xl font-bold text-gray-900 dark:text-white">Gimnasio</h1>
          </div>
          <p className="text-gray-600 dark:text-gray-300">
            Organiza tus entrenamientos por rutinas o registra ejercicios individuales
          </p>
        </div>

        {/* Pestañas de navegación */}
        <div className="flex gap-2 mb-6 flex-wrap">
          <button
            onClick={() => setViewMode("routines")}
            className={`px-4 py-2 rounded-lg font-medium transition-colors ${
              viewMode === "routines" || viewMode === "routine-detail" || viewMode === "create-routine"
                ? "bg-blue-600 text-white"
                : "bg-gray-100 dark:bg-gray-700 text-gray-700 dark:text-gray-300 hover:bg-gray-200 dark:hover:bg-gray-600"
            }`}
          >
            Rutinas
          </button>
          <button
            onClick={() => setViewMode("individual")}
            className={`px-4 py-2 rounded-lg font-medium transition-colors ${
              viewMode === "individual" ? "bg-blue-600 text-white" : "bg-gray-100 dark:bg-gray-700 text-gray-700 dark:text-gray-300 hover:bg-gray-200 dark:hover:bg-gray-600"
            }`}
          >
            Ejercicios Individuales
          </button>
          <button
            onClick={() => setViewMode("history")}
            className={`px-4 py-2 rounded-lg font-medium transition-colors ${
              viewMode === "history" ? "bg-blue-600 text-white" : "bg-gray-100 dark:bg-gray-700 text-gray-700 dark:text-gray-300 hover:bg-gray-200 dark:hover:bg-gray-600"
            }`}
          >
            Historial
          </button>
          <button
            onClick={() => setViewMode("metrics")}
            className={`px-4 py-2 rounded-lg font-medium transition-colors ${
              viewMode === "metrics" ? "bg-blue-600 text-white" : "bg-gray-100 dark:bg-gray-700 text-gray-700 dark:text-gray-300 hover:bg-gray-200 dark:hover:bg-gray-600"
            }`}
          >
            Métricas
          </button>
        </div>

        {/* Renderizado condicional de componentes */}
        <div className="grid gap-6">
          {viewMode === "routines" && (
            <RoutineList
              refreshTrigger={refreshTrigger}
              onViewRoutine={handleViewRoutine}
              onCreateRoutine={handleCreateRoutine}
            />
          )}

          {viewMode === "create-routine" && (
            <RoutineForm 
              onRoutineCreated={handleRoutineCreated} 
              onCancel={() => setViewMode("routines")} 
            />
          )}

          {viewMode === "routine-detail" && selectedRoutine && (
            <RoutineDetail
              routineId={selectedRoutine.id}
              routineName={selectedRoutine.name}
              onBack={handleBackToRoutines}
            />
          )}

          {viewMode === "individual" && (
            <>
              <WorkoutForm
                onWorkoutAdded={handleWorkoutAdded}
                editWorkout={editingWorkout}
                onEditComplete={handleEditComplete}
              />
              <WorkoutList 
                refreshTrigger={refreshTrigger} 
                onEditWorkout={handleEditWorkout} 
              />
            </>
          )}

          {viewMode === "history" && <ExerciseHistory />}
          {viewMode === "metrics" && <GymMetrics />}
        </div>
      </div>
    </div>
  )
}
\end{lstlisting}

\textbf{Características del componente:}
\begin{itemize}
    \item \textbf{Estado centralizado}: Maneja todos los estados de la aplicación
    \item \textbf{Navegación por pestañas}: Interfaz intuitiva para cambiar entre vistas
    \item \textbf{Renderizado condicional}: Muestra diferentes componentes según el modo
    \item \textbf{Handlers de eventos}: Gestiona todas las interacciones del usuario
    \item \textbf{TypeScript}: Tipado estricto para mayor seguridad
\end{itemize}

\subsection{Formulario de Entrenamiento - WorkoutForm}

Componente complejo que maneja la creación y edición de ejercicios.

\begin{lstlisting}[language=typescript, caption=components/gym/workout-form.tsx - Estructura principal]
"use client"

import { useState, useEffect } from "react"
import { Button } from "@/components/ui/button"
import { Input } from "@/components/ui/input"
import { Card, CardContent, CardDescription, CardHeader, CardTitle } from "@/components/ui/card"
import { Label } from "@/components/ui/label"
import { Plus, Loader2, Edit, Dumbbell } from "lucide-react"
import { createWorkout, updateWorkout } from "@/lib/gym-actions"
import { addExerciseToRoutine, updateExerciseInRoutine } from "@/lib/routine-actions"
import { useActionState } from "react"
import ExerciseSelectorModal from "./exercise-selector-modal"

// Interfaces TypeScript
interface Workout {
  id: string
  exercise_name: string
  weight_kg: number | null
  repetitions: number | null
  sets: number | null
  image_url: string | null
  created_at: string
}

interface GymExercise {
  id: string
  name: string
  category: string
  description: string | null
  image_url: string | null
}

interface WorkoutFormProps {
  onWorkoutAdded?: () => void
  editWorkout?: Workout | null
  onEditComplete?: () => void
  routineId?: string
}

export default function WorkoutForm({ 
  onWorkoutAdded, 
  editWorkout, 
  onEditComplete, 
  routineId 
}: WorkoutFormProps) {
  // Estados del componente
  const [state, formAction] = useActionState(editWorkout ? updateWorkout : createWorkout, null)
  const [isOpen, setIsOpen] = useState(false)
  const [showExerciseModal, setShowExerciseModal] = useState(false)
  const [selectedExercise, setSelectedExercise] = useState<GymExercise | null>(null)
  const [customExerciseName, setCustomExerciseName] = useState("")

  // Efecto para abrir formulario cuando se edita
  useEffect(() => {
    if (editWorkout) {
      setIsOpen(true)
    }
  }, [editWorkout])

  // Handler para selección de ejercicio
  const handleExerciseSelect = (exercise: GymExercise) => {
    if (exercise.id === "custom") {
      setSelectedExercise(null)
      setCustomExerciseName("")
    } else {
      setSelectedExercise(exercise)
      setCustomExerciseName("")
    }
  }

  // Handler principal de envío
  const handleSubmit = async (formData: FormData) => {
    const exerciseName = selectedExercise?.name || customExerciseName

    if (!exerciseName) {
      return
    }

    const imageUrl = selectedExercise?.image_url || formData.get("image_url")?.toString() || null

    // Preparar FormData
    const newFormData = new FormData()
    newFormData.append("exercise_name", exerciseName)
    newFormData.append("weight_kg", formData.get("weight_kg")?.toString() || "")
    newFormData.append("repetitions", formData.get("repetitions")?.toString() || "")
    newFormData.append("sets", formData.get("sets")?.toString() || "")
    newFormData.append("image_url", imageUrl || "")

    if (editWorkout) {
      newFormData.append("id", editWorkout.id)
    }

    // Lógica para rutinas vs ejercicios individuales
    if (routineId) {
      const exerciseData = {
        exercise_name: exerciseName,
        weight_kg: formData.get("weight_kg") ? 
          Number.parseFloat(formData.get("weight_kg")?.toString() || "0") : null,
        repetitions: formData.get("repetitions") ? 
          Number.parseInt(formData.get("repetitions")?.toString() || "0") : null,
        sets: formData.get("sets") ? 
          Number.parseInt(formData.get("sets")?.toString() || "0") : null,
        image_url: imageUrl,
      }

      let result
      if (editWorkout) {
        result = await updateExerciseInRoutine(editWorkout.id, exerciseData)
      } else {
        result = await addExerciseToRoutine(routineId, exerciseData)
      }

      if (result?.success) {
        setIsOpen(false)
        setSelectedExercise(null)
        setCustomExerciseName("")
        if (editWorkout) {
          onEditComplete?.()
        } else {
          onWorkoutAdded?.()
        }
      }
    } else {
      // Ejercicio individual
      const result = await formAction(newFormData)
      if (result?.success) {
        setIsOpen(false)
        setSelectedExercise(null)
        setCustomExerciseName("")
        if (editWorkout) {
          onEditComplete?.()
        } else {
          onWorkoutAdded?.()
        }
      }
    }
  }

  // Handler para cancelar
  const handleCancel = () => {
    setIsOpen(false)
    setSelectedExercise(null)
    setCustomExerciseName("")
    if (editWorkout) {
      onEditComplete?.()
    }
  }

  // Renderizar botón si no está abierto
  if (!isOpen && !editWorkout) {
    return (
      <Button onClick={() => setIsOpen(true)} className="w-full bg-blue-600 hover:bg-blue-700">
        <Plus className="h-4 w-4 mr-2" />
        {routineId ? "Agregar Ejercicio a la Rutina" : "Agregar Ejercicio"}
      </Button>
    )
  }

  return (
    <>
      {/* Modal selector de ejercicios */}
      <ExerciseSelectorModal
        open={showExerciseModal}
        onOpenChange={setShowExerciseModal}
        onSelectExercise={handleExerciseSelect}
      />

      {/* Formulario principal */}
      <Card>
        <CardHeader>
          <CardTitle className="flex items-center gap-2">
            {editWorkout ? (
              <>
                <Edit className="h-5 w-5" />
                Editar Ejercicio
              </>
            ) : (
              <>
                <Plus className="h-5 w-5" />
                Nuevo Ejercicio
              </>
            )}
          </CardTitle>
          <CardDescription>
            {editWorkout
              ? "Modifica los datos de tu entrenamiento"
              : routineId
                ? "Agrega un ejercicio a esta rutina"
                : "Registra tu entrenamiento de gimnasio"}
          </CardDescription>
        </CardHeader>
        <CardContent>
          <form action={handleSubmit} className="space-y-4">
            {/* Mostrar errores */}
            {state?.error && (
              <div className="bg-red-50 border border-red-200 text-red-700 px-4 py-3 rounded text-sm">
                {state.error}
              </div>
            )}

            <div className="grid grid-cols-1 gap-4">
              {/* Selector de ejercicio (solo para nuevos) */}
              {!editWorkout && (
                <div className="space-y-2">
                  <Label>Ejercicio</Label>
                  {selectedExercise ? (
                    <div className="flex items-center gap-3 p-3 border rounded-lg bg-accent/50">
                      {selectedExercise.image_url ? (
                        <img
                          src={selectedExercise.image_url || "/placeholder.svg"}
                          alt={selectedExercise.name}
                          className="w-12 h-12 rounded object-cover"
                        />
                      ) : (
                        <div className="w-12 h-12 rounded bg-muted flex items-center justify-center">
                          <Dumbbell className="h-6 w-6 text-muted-foreground" />
                        </div>
                      )}
                      <div className="flex-1">
                        <div className="font-medium">{selectedExercise.name}</div>
                        <div className="text-xs text-muted-foreground">{selectedExercise.category}</div>
                      </div>
                      <Button type="button" variant="ghost" size="sm" onClick={() => setShowExerciseModal(true)}>
                        Cambiar
                      </Button>
                    </div>
                  ) : customExerciseName ? (
                    <div className="flex items-center gap-3 p-3 border rounded-lg bg-accent/50">
                      <div className="w-12 h-12 rounded bg-muted flex items-center justify-center">
                        <Dumbbell className="h-6 w-6 text-muted-foreground" />
                      </div>
                      <div className="flex-1">
                        <div className="font-medium">{customExerciseName}</div>
                        <div className="text-xs text-muted-foreground">Ejercicio personalizado</div>
                      </div>
                      <Button type="button" variant="ghost" size="sm" onClick={() => setShowExerciseModal(true)}>
                        Cambiar
                      </Button>
                    </div>
                  ) : (
                    <Button
                      type="button"
                      variant="outline"
                      className="w-full h-auto py-4 bg-transparent"
                      onClick={() => setShowExerciseModal(true)}
                    >
                      <Dumbbell className="h-5 w-5 mr-2" />
                      Seleccionar ejercicio del catálogo
                    </Button>
                  )}

                  {/* Campo para ejercicio personalizado */}
                  {!selectedExercise && (
                    <div className="space-y-2">
                      <Label htmlFor="custom_exercise_name">O escribe un ejercicio personalizado</Label>
                      <Input
                        id="custom_exercise_name"
                        name="custom_exercise_name"
                        placeholder="Ej: Press de banca, Sentadillas..."
                        value={customExerciseName}
                        onChange={(e) => setCustomExerciseName(e.target.value)}
                      />
                    </div>
                  )}
                </div>
              )}

              {/* Mostrar nombre del ejercicio en modo edición */}
              {editWorkout && (
                <div className="space-y-2">
                  <Label>Ejercicio</Label>
                  <Input value={editWorkout.exercise_name} disabled />
                </div>
              )}

              {/* Campos de datos del ejercicio */}
              <div className="grid grid-cols-2 gap-4">
                <div className="space-y-2">
                  <Label htmlFor="weight_kg">Peso (kg)</Label>
                  <Input
                    id="weight_kg"
                    name="weight_kg"
                    type="number"
                    step="0.5"
                    min="0"
                    placeholder="80"
                    defaultValue={editWorkout?.weight_kg ?? ""}
                  />
                </div>
                <div className="space-y-2">
                  <Label htmlFor="repetitions">Repeticiones</Label>
                  <Input
                    id="repetitions"
                    name="repetitions"
                    type="number"
                    min="1"
                    placeholder="12"
                    defaultValue={editWorkout?.repetitions ?? ""}
                  />
                </div>
              </div>
              <div className="space-y-2">
                <Label htmlFor="sets">Series</Label>
                <Input
                  id="sets"
                  name="sets"
                  type="number"
                  min="1"
                  placeholder="3"
                  defaultValue={editWorkout?.sets ?? ""}
                />
              </div>

              {/* Campo para URL de imagen (solo ejercicios personalizados) */}
              {!editWorkout && !selectedExercise && customExerciseName && (
                <div className="space-y-2">
                  <Label htmlFor="image_url">URL de la Imagen (opcional)</Label>
                  <Input
                    id="image_url"
                    name="image_url"
                    type="url"
                    placeholder="https://ejemplo.com/imagen-ejercicio.jpg"
                  />
                  <p className="text-xs text-gray-500">Solo para ejercicios personalizados</p>
                </div>
              )}
            </div>

            {/* Botones de acción */}
            <div className="flex gap-2">
              <Button type="submit" className="flex-1 bg-green-600 hover:bg-green-700">
                <Loader2 className="h-4 w-4 mr-2 animate-spin hidden" />
                {editWorkout ? "Actualizar Ejercicio" : "Guardar Ejercicio"}
              </Button>
              <Button type="button" variant="outline" onClick={handleCancel}>
                Cancelar
              </Button>
            </div>
          </form>
        </CardContent>
      </Card>
    </>
  )
}
\end{lstlisting}

\textbf{Características del formulario:}
\begin{itemize}
    \item \textbf{Selector de ejercicios}: Modal para elegir del catálogo o crear personalizado
    \item \textbf{Validación en tiempo real}: Verifica datos antes de enviar
    \item \textbf{Modo edición}: Reutiliza el mismo componente para editar
    \item \textbf{Soporte para rutinas}: Funciona tanto para ejercicios individuales como rutinas
    \item \textbf{Manejo de imágenes}: Soporte para URLs de imágenes
\end{itemize}

\subsection{Lista de Entrenamientos - WorkoutList}

Componente que muestra y gestiona la lista de entrenamientos del usuario.

\begin{lstlisting}[language=typescript, caption=components/gym/workout-list.tsx - Estructura principal]
"use client"

import { useState, useEffect } from "react"
import { Card, CardContent } from "@/components/ui/card"
import { Button } from "@/components/ui/button"
import { Badge } from "@/components/ui/badge"
import { Trash2, Calendar, Weight, RotateCcw, Hash, Edit } from "lucide-react"
import { deleteWorkout, getWorkouts } from "@/lib/gym-actions"
import { format } from "date-fns"
import { es } from "date-fns/locale"

interface Workout {
  id: string
  exercise_name: string
  weight_kg: number | null
  repetitions: number | null
  sets: number | null
  image_url: string | null
  created_at: string
}

interface WorkoutListProps {
  refreshTrigger?: number
  onEditWorkout?: (workout: Workout) => void
}

export default function WorkoutList({ refreshTrigger, onEditWorkout }: WorkoutListProps) {
  const [workouts, setWorkouts] = useState<Workout[]>([])
  const [loading, setLoading] = useState(true)

  // Cargar entrenamientos
  const loadWorkouts = async () => {
    try {
      const data = await getWorkouts()
      setWorkouts(data || [])
    } catch (error) {
      console.error("Error loading workouts:", error)
    } finally {
      setLoading(false)
    }
  }

  // Efecto para recargar cuando cambia el trigger
  useEffect(() => {
    loadWorkouts()
  }, [refreshTrigger])

  // Handler para eliminar entrenamiento
  const handleDelete = async (id: string) => {
    if (confirm("¿Estás seguro de que quieres eliminar este ejercicio?")) {
      const result = await deleteWorkout(id)
      if (result?.success) {
        loadWorkouts()
      }
    }
  }

  // Handler para editar entrenamiento
  const handleEdit = (workout: Workout) => {
    onEditWorkout?.(workout)
  }

  // Estado de carga
  if (loading) {
    return (
      <Card>
        <CardContent className="p-6">
          <div className="text-center text-gray-500">Cargando entrenamientos...</div>
        </CardContent>
      </Card>
    )
  }

  // Estado vacío
  if (workouts.length === 0) {
    return (
      <Card>
        <CardContent className="p-6">
          <div className="text-center text-gray-500">
            <Weight className="h-12 w-12 mx-auto mb-4 text-gray-300" />
            <p>No hay entrenamientos registrados</p>
            <p className="text-sm">Agrega tu primer ejercicio para comenzar</p>
          </div>
        </CardContent>
      </Card>
    )
  }

  return (
    <div className="space-y-4">
      <h3 className="text-lg font-semibold">Historial de Entrenamientos</h3>
      {workouts.map((workout) => (
        <Card key={workout.id}>
          <CardContent className="p-4">
            <div className="flex justify-between items-start">
              <div className="flex gap-4 flex-1">
                {/* Imagen del ejercicio */}
                {workout.image_url && (
                  <div className="flex-shrink-0">
                    <img
                      src={workout.image_url || "/placeholder.svg"}
                      alt={workout.exercise_name}
                      className="w-16 h-16 object-cover rounded-lg border"
                      onError={(e) => {
                        const target = e.target as HTMLImageElement
                        target.style.display = "none"
                      }}
                    />
                  </div>
                )}
                
                {/* Información del ejercicio */}
                <div className="flex-1">
                  <h4 className="font-semibold text-lg">{workout.exercise_name}</h4>
                  
                  {/* Badges con métricas */}
                  <div className="flex flex-wrap gap-2 mt-2">
                    {workout.weight_kg && (
                      <Badge variant="secondary" className="flex items-center gap-1">
                        <Weight className="h-3 w-3" />
                        {workout.weight_kg} kg
                      </Badge>
                    )}
                    {workout.repetitions && (
                      <Badge variant="secondary" className="flex items-center gap-1">
                        <RotateCcw className="h-3 w-3" />
                        {workout.repetitions} reps
                      </Badge>
                    )}
                    {workout.sets && (
                      <Badge variant="secondary" className="flex items-center gap-1">
                        <Hash className="h-3 w-3" />
                        {workout.sets} series
                      </Badge>
                    )}
                  </div>
                  
                  {/* Fecha y hora */}
                  <div className="flex items-center gap-1 mt-2 text-sm text-gray-500">
                    <Calendar className="h-3 w-3" />
                    {format(new Date(workout.created_at), "PPP 'a las' HH:mm", { locale: es })}
                  </div>
                </div>
              </div>
              
              {/* Botones de acción */}
              <div className="flex gap-2">
                <Button
                  variant="outline"
                  size="sm"
                  onClick={() => handleEdit(workout)}
                  className="text-blue-600 hover:text-blue-700"
                >
                  <Edit className="h-4 w-4" />
                </Button>
                <Button
                  variant="outline"
                  size="sm"
                  onClick={() => handleDelete(workout.id)}
                  className="text-red-600 hover:text-red-700"
                >
                  <Trash2 className="h-4 w-4" />
                </Button>
              </div>
            </div>
          </CardContent>
        </Card>
      ))}
    </div>
  )
}
\end{lstlisting}

\section{Ejemplos Prácticos de Uso}

\subsection{Ejemplos de Inserción a Base de Datos}

\subsubsection{Crear un Entrenamiento Individual}

\begin{lstlisting}[style=sqlstyle, caption=Ejemplo de inserción en gym_workouts]
-- Insertar un entrenamiento de press de banca
INSERT INTO public.gym_workouts (
    user_id,
    exercise_name,
    weight_kg,
    repetitions,
    sets,
    image_url
) VALUES (
    '123e4567-e89b-12d3-a456-426614174000', -- UUID del usuario
    'Press de Banca',
    80.00,
    12,
    3,
    'https://ejemplo.com/press-banca.jpg'
);

-- Insertar un entrenamiento de sentadillas
INSERT INTO public.gym_workouts (
    user_id,
    exercise_name,
    weight_kg,
    repetitions,
    sets
) VALUES (
    '123e4567-e89b-12d3-a456-426614174000',
    'Sentadillas',
    100.00,
    10,
    4
);
\end{lstlisting}

\subsubsection{Crear una Rutina Completa}

\begin{lstlisting}[style=sqlstyle, caption=Ejemplo de creación de rutina completa]
-- 1. Crear la rutina
INSERT INTO public.routines (
    user_id,
    name,
    description
) VALUES (
    '123e4567-e89b-12d3-a456-426614174000',
    'Rutina de Pecho y Tríceps',
    'Rutina enfocada en el desarrollo del pecho y tríceps, ideal para principiantes'
);

-- Obtener el ID de la rutina recién creada
-- (En la aplicación esto se maneja automáticamente)

-- 2. Agregar ejercicios a la rutina
INSERT INTO public.routine_exercises (
    routine_id,
    exercise_name,
    weight,
    repetitions,
    sets,
    order_index
) VALUES 
    ('456e7890-e89b-12d3-a456-426614174001', 'Press de Banca', 80.00, 12, 3, 0),
    ('456e7890-e89b-12d3-a456-426614174001', 'Aperturas con Mancuernas', 25.00, 15, 3, 1),
    ('456e7890-e89b-12d3-a456-426614174001', 'Press Francés', 30.00, 12, 3, 2),
    ('456e7890-e89b-12d3-a456-426614174001', 'Fondos para Tríceps', 0.00, 10, 3, 3);
\end{lstlisting}

\subsubsection{Consultas Útiles}

\begin{lstlisting}[style=sqlstyle, caption=Consultas útiles para análisis]
-- Obtener todos los entrenamientos de un usuario con detalles
SELECT 
    gw.exercise_name,
    gw.weight_kg,
    gw.repetitions,
    gw.sets,
    gw.created_at,
    (gw.weight_kg * gw.repetitions * gw.sets) as volumen_total
FROM public.gym_workouts gw
WHERE gw.user_id = '123e4567-e89b-12d3-a456-426614174000'
ORDER BY gw.created_at DESC;

-- Obtener progreso de un ejercicio específico
SELECT 
    exercise_name,
    weight_kg,
    repetitions,
    sets,
    created_at,
    (weight_kg * repetitions * sets) as volumen
FROM public.gym_workouts
WHERE user_id = '123e4567-e89b-12d3-a456-426614174000'
  AND exercise_name = 'Press de Banca'
ORDER BY created_at ASC;

-- Obtener rutinas con conteo de ejercicios
SELECT 
    r.name,
    r.description,
    r.created_at,
    COUNT(re.id) as total_ejercicios
FROM public.routines r
LEFT JOIN public.routine_exercises re ON r.id = re.routine_id
WHERE r.user_id = '123e4567-e89b-12d3-a456-426614174000'
GROUP BY r.id, r.name, r.description, r.created_at
ORDER BY r.created_at DESC;

-- Obtener ejercicios de una rutina específica en orden
SELECT 
    re.exercise_name,
    re.weight,
    re.repetitions,
    re.sets,
    re.order_index
FROM public.routine_exercises re
WHERE re.routine_id = '456e7890-e89b-12d3-a456-426614174001'
ORDER BY re.order_index ASC;
\end{lstlisting}

\subsection{Flujos de Datos Completos}

\subsubsection{Flujo de Creación de Entrenamiento}

\begin{enumerate}
    \item \textbf{Usuario completa formulario} en \texttt{WorkoutForm}
    \item \textbf{Validación en cliente} de campos requeridos
    \item \textbf{Envío a Server Action} \texttt{createWorkout}
    \item \textbf{Verificación de autenticación} en servidor
    \item \textbf{Sanitización de datos} (conversión de tipos)
    \item \textbf{Inserción en base de datos} tabla \texttt{gym\_workouts}
    \item \textbf{Revalidación de cache} con \texttt{revalidatePath}
    \item \textbf{Actualización de UI} con nuevo entrenamiento
\end{enumerate}

\subsubsection{Flujo de Gestión de Rutinas}

\begin{enumerate}
    \item \textbf{Creación de rutina} con \texttt{createRoutine}
    \item \textbf{Inserción en tabla} \texttt{routines}
    \item \textbf{Agregar ejercicios} con \texttt{addExerciseToRoutine}
    \item \textbf{Inserción en tabla} \texttt{routine\_exercises} con \texttt{order\_index}
    \item \textbf{Visualización de rutina} con \texttt{getRoutineExercises}
    \item \textbf{Ejecución de rutina} (registro de entrenamientos individuales)
\end{enumerate}

\section{Características Avanzadas}

\subsection{Selector de Ejercicios}

El módulo incluye un selector modal avanzado que permite:

\begin{itemize}
    \item \textbf{Catálogo de ejercicios}: Acceso a ejercicios administrados
    \item \textbf{Filtrado por categoría}: Pecho, Bíceps, Tríceps, etc.
    \item \textbf{Búsqueda de ejercicios}: Filtrado por nombre
    \item \textbf{Ejercicios personalizados}: Creación de ejercicios únicos
    \item \textbf{Imágenes de ejercicios}: Visualización de técnicas
\end{itemize}

\subsection{Métricas y Estadísticas}

El componente \texttt{GymMetrics} proporciona:

\begin{itemize}
    \item \textbf{Gráficos de progreso}: Evolución del peso a lo largo del tiempo
    \item \textbf{Estadísticas de volumen}: Cálculo de volumen total (peso × reps × series)
    \item \textbf{Tendencias de entrenamiento}: Frecuencia y consistencia
    \item \textbf{Comparativas temporales}: Progreso mensual/semanal
    \item \textbf{Análisis por ejercicio}: Progreso específico por ejercicio
\end{itemize}

\subsection{Historial de Ejercicios}

El componente \texttt{ExerciseHistory} permite:

\begin{itemize}
    \item \textbf{Historial completo}: Todos los entrenamientos registrados
    \item \textbf{Filtrado por ejercicio}: Ver progreso de un ejercicio específico
    \item \textbf{Análisis temporal}: Progreso a lo largo del tiempo
    \item \textbf{Exportación de datos}: Para análisis externo
    \item \textbf{Búsqueda avanzada}: Filtros por fecha, ejercicio, peso
\end{itemize}

\section{Mejores Prácticas de Desarrollo}

\subsection{Seguridad}

\begin{itemize}
    \item \textbf{Validación de entrada}: Siempre validar datos en Server Actions
    \item \textbf{Autenticación}: Verificar usuario en cada operación
    \item \textbf{RLS}: Usar Row Level Security en todas las tablas
    \item \textbf{Sanitización}: Limpiar y convertir datos de entrada
    \item \textbf{Logs de seguridad}: Registrar operaciones sensibles
\end{itemize}

\subsection{Performance}

\begin{itemize}
    \item \textbf{Índices de base de datos}: Optimizar consultas frecuentes
    \item \textbf{Paginación}: Implementar para listas largas
    \item \textbf{Cache}: Usar revalidatePath para actualizar cache
    \item \textbf{Lazy loading}: Cargar componentes bajo demanda
    \item \textbf{Optimización de imágenes}: Comprimir y optimizar imágenes
\end{itemize}

\subsection{Mantenibilidad}

\begin{itemize}
    \item \textbf{TypeScript}: Usar tipado estricto en todos los componentes
    \item \textbf{Interfaces}: Definir interfaces claras para props
    \item \textbf{Separación de responsabilidades}: Lógica en Server Actions
    \item \textbf{Reutilización}: Componentes modulares y reutilizables
    \item \textbf{Documentación}: Comentar código complejo
\end{itemize}

\section{Solución de Problemas}

\subsection{Problemas Comunes}

\subsubsection{Error: "Tabla no existe"}

\textbf{Síntomas}: Error al intentar insertar o consultar datos
\textbf{Causa}: Las tablas no han sido creadas en la base de datos
\textbf{Solución}: Ejecutar los scripts SQL en orden:
\begin{enumerate}
    \item \texttt{01-create-database-schema.sql}
    \item \texttt{01-create-user-schema.sql}
    \item \texttt{08-verify-routines-tables.sql}
    \item \texttt{26-create-gym-exercises-table.sql}
\end{enumerate}

\subsubsection{Error: "Usuario no autenticado"}

\textbf{Síntomas}: Server Actions retornan error de autenticación
\textbf{Causa}: Usuario no está logueado o sesión expirada
\textbf{Solución}: Verificar estado de autenticación y redirigir a login

\subsubsection{Error: "Política RLS violada"}

\textbf{Síntomas}: Error al acceder a datos de otros usuarios
\textbf{Causa}: Políticas RLS mal configuradas
\textbf{Solución}: Verificar y corregir políticas RLS

\subsection{Debugging}

\subsubsection{Logs del Cliente}

\begin{lstlisting}[language=javascript, caption=Debugging en cliente]
// Habilitar logs detallados
localStorage.setItem('debug', 'true');

// Verificar estado de autenticación
console.log('User:', user);
console.log('Session:', session);

// Verificar datos de entrenamientos
console.log('Workouts:', workouts);
\end{lstlisting}

\subsubsection{Logs del Servidor}



\textbf{Características destacadas:}
\begin{itemize}
    \item Seguridad robusta con RLS
    \item Interfaz de usuario intuitiva
    \item Funcionalidades avanzadas de análisis
    \item Código bien documentado y tipado
\end{itemize}

Para contribuir al desarrollo del módulo:
\begin{enumerate}
    \item Seguir las convenciones establecidas
    \item Implementar tests apropiados
    \item Documentar cambios significativos
    \item Mantener la compatibilidad con la base de datos
    \item Respetar las políticas de seguridad
\end{enumerate}

\end{document}
