\documentclass[12pt,a4paper]{article}
\usepackage[utf8]{inputenc}
\usepackage[spanish]{babel}
\usepackage{geometry}
\usepackage{graphicx}
\usepackage{listings}
\usepackage{xcolor}
\usepackage{hyperref}
\usepackage{fancyhdr}
\usepackage{titlesec}
\usepackage{enumitem}
\usepackage{booktabs}
\usepackage{array}

\geometry{margin=2.5cm}
\pagestyle{fancy}
\fancyhf{}
\fancyhead[L]{Manual del Programador - FitTrack}
\fancyhead[R]{Módulo de Administrador}
\fancyfoot[C]{\thepage}

\definecolor{codegreen}{rgb}{0,0.6,0}
\definecolor{codegray}{rgb}{0.5,0.5,0.5}
\definecolor{codepurple}{rgb}{0.58,0,0.82}
\definecolor{backcolour}{rgb}{0.95,0.95,0.92}

\lstset{
	backgroundcolor=\color{backcolour},   
	commentstyle=\color{codegreen},
	keywordstyle=\color{blue},
	numberstyle=\tiny\color{codegray},
	stringstyle=\color{codepurple},
	basicstyle=\ttfamily\footnotesize,
	breakatwhitespace=false,         
	breaklines=true,                 
	captionpos=b,                    
	keepspaces=true,                 
	numbers=left,                    
	numbersep=5pt,                  
	showspaces=false,                
	showstringspaces=false,
	showtabs=false,                  
	tabsize=2,
	language=SQL
}

\title{\textbf{MÓDULO DE ADMINISTRADOR\\FitTrack}}
\author{Manual del Programador}
\date{\today}

\begin{document}
	
	\maketitle
	
	\tableofcontents
	\newpage
	
	\section{Introducción al Módulo de Administrador}
	
	El módulo de administrador en FitTrack es un sistema completo de gestión que permite a usuarios con privilegios administrativos controlar y supervisar toda la aplicación. Este módulo está diseñado con un enfoque de seguridad robusto, utilizando Row Level Security (RLS) de Supabase y un sistema de roles bien definido.
	
	\subsection{Características Principales}
	
	\begin{itemize}[leftmargin=2cm]
		\item \textbf{Gestión de Usuarios}: Control completo sobre roles, permisos y estado de usuarios
		\item \textbf{Administración de Ejercicios}: Gestión del catálogo de ejercicios disponibles
		\item \textbf{Sistema de Roles}: Implementación de roles de usuario, profesional y administrador
		\item \textbf{Seguridad Avanzada}: Protección mediante RLS y funciones de seguridad
		\item \textbf{Interfaz Intuitiva}: Dashboard moderno con componentes React optimizados
	\end{itemize}
	
	\section{Arquitectura del Sistema}
	
	\subsection{Estructura de Archivos}
	
	El módulo de administrador se organiza en la siguiente estructura:
	
	\begin{lstlisting}[language=bash, caption=Estructura del módulo de administrador]
		app/
		|-- admin/
		|   |-- page.tsx                 # Página principal del admin
		|   |-- exercises/
		|       |-- page.tsx            # Gestión de ejercicios
		components/
		|-- admin/
		|   |-- admin-dashboard.tsx     # Dashboard principal
		|   |-- exercise-management.tsx # Gestión de ejercicios
		lib/
		|-- admin-actions.ts            # Acciones del servidor
		scripts/
		|-- 11-create-messaging-system.sql
		|-- 12-create-admin-user.sql
		|-- 16-fix-user-roles-rls.sql
		|-- 17-add-role-management-functions.sql
		|-- 18-create-get-all-users-function.sql
		|-- 19-add-is-professional-field.sql
		|-- 26-create-gym-exercises-table.sql
	\end{lstlisting}
	
	\subsection{Base de Datos}
	
	El sistema utiliza las siguientes tablas principales:
	
	\subsubsection{Tabla user\_roles}
	\begin{lstlisting}[language=SQL, caption=Estructura de la tabla user\_roles]
		CREATE TABLE user_roles (
		id UUID PRIMARY KEY DEFAULT gen_random_uuid(),
		user_id UUID REFERENCES auth.users(id) ON DELETE CASCADE,
		role TEXT NOT NULL CHECK (role IN ('user', 'admin')),
		is_active BOOLEAN DEFAULT true,
		is_professional BOOLEAN DEFAULT false,
		approved_by UUID REFERENCES auth.users(id),
		approved_at TIMESTAMPTZ,
		created_at TIMESTAMPTZ DEFAULT NOW(),
		updated_at TIMESTAMPTZ DEFAULT NOW()
		);
	\end{lstlisting}
	
	\subsubsection{Tabla gym\_exercises}
	\begin{lstlisting}[language=SQL, caption=Estructura de la tabla gym\_exercises]
		CREATE TABLE gym_exercises (
		id UUID PRIMARY KEY DEFAULT gen_random_uuid(),
		name TEXT NOT NULL,
		category TEXT NOT NULL CHECK (category IN ('Pecho', 'B\'iceps', 'Tr\'iceps', 'Hombros', 'Pierna', 'Espalda', 'Otros')),
		description TEXT,
		image_url TEXT,
		created_at TIMESTAMPTZ DEFAULT NOW(),
		updated_at TIMESTAMPTZ DEFAULT NOW()
		);
	\end{lstlisting}
	
	\section{Sistema de Autenticación y Autorización}
	
	\subsection{Verificación de Privilegios de Administrador}
	
	La función \texttt{isAdmin()} es el núcleo del sistema de autorización:
	
	\begin{lstlisting}[language=JavaScript, caption=Función isAdmin en admin-actions.ts]
		export async function isAdmin() {
			try {
				const supabase = await createClient()
				const { data: { user }, error: userError } = await supabase.auth.getUser()
				
				if (userError || !user) {
					return false
				}
				
				const { data, error } = await supabase
				.from("user_roles")
				.select("role")
				.eq("user_id", user.id)
				.maybeSingle()
				
				if (error || !data) {
					return false
				}
				
				return data.role === "admin"
			} catch (error) {
				console.error("[v0] Exception in isAdmin:", error)
				return false
			}
		}
	\end{lstlisting}
	
	\subsection{Protección de Rutas}
	
	Cada página de administración implementa protección de acceso:
	
	\begin{lstlisting}[language=JavaScript, caption=Protección de rutas en page.tsx]
		export default async function AdminPage() {
			const admin = await isAdmin()
			
			if (!admin) {
				redirect("/")
			}
			// ... resto del componente
		}
	\end{lstlisting}
	
	\section{Funcionalidades del Dashboard Principal}
	
	\subsection{Componente AdminDashboard}
	
	El componente principal del dashboard ofrece:
	
	\begin{enumerate}
		\item \textbf{Estadísticas del Sistema}:
		\begin{itemize}
			\item Total de usuarios registrados
			\item Número de profesionales registrados
			\item Cantidad de administradores
		\end{itemize}
		
		\item \textbf{Gestión de Usuarios}:
		\begin{itemize}
			\item Lista completa de usuarios con información detallada
			\item Cambio de roles (Usuario/Admin)
			\item Activación/desactivación de usuarios
			\item Marcado de usuarios como profesionales
			\item Búsqueda y filtrado de usuarios
		\end{itemize}
		
		\item \textbf{Interfaz de Usuario}:
		\begin{itemize}
			\item Diseño responsivo con Tailwind CSS
			\item Componentes de UI modernos (shadcn/ui)
			\item Notificaciones toast para feedback
			\item Estados de carga durante operaciones
		\end{itemize}
	\end{enumerate}
	
	\subsection{Estados y Gestión de Datos}
	
	El dashboard utiliza React hooks para el manejo de estado:
	
	\begin{lstlisting}[language=JavaScript, caption=Estados del componente AdminDashboard]
		const [users, setUsers] = useState(initialUsers)
		const [loading, setLoading] = useState<string | null>(null)
		const [searchQuery, setSearchQuery] = useState("")
		const { toast } = useToast()
	\end{lstlisting}
	
	\subsection{Funciones de Gestión de Usuarios}
	
	\subsubsection{Cambio de Roles}
	\begin{lstlisting}[language=JavaScript, caption=Función handleRoleChange]
		const handleRoleChange = async (userId: string, role: string) => {
			setLoading(userId)
			const user = users.find((u) => u.id === userId)
			if (!user) return
			
			const result = await updateUserRole(userId, role, user.is_active, user.is_professional)
			
			if (result.error) {
				toast({
					title: "Error",
					description: result.error,
					variant: "destructive",
				})
			} else {
				toast({
					title: "Éxito",
					description: "Rol actualizado correctamente",
				})
				setUsers(users.map((u) => (u.id === userId ? { ...u, role } : u)))
			}
			setLoading(null)
		}
	\end{lstlisting}
	
	\section{Gestión de Ejercicios}
	
	\subsection{Componente ExerciseManagement}
	
	El sistema de gestión de ejercicios permite a los administradores:
	
	\begin{enumerate}
		\item \textbf{Crear Nuevos Ejercicios}:
		\begin{itemize}
			\item Nombre del ejercicio
			\item Categoría (Pecho, B\'iceps, Tr\'iceps, Hombros, Pierna, Espalda, Otros)
			\item Descripción opcional
			\item Imagen del ejercicio (URL externa o subida de archivo)
		\end{itemize}
		
		\item \textbf{Editar Ejercicios Existentes}:
		\begin{itemize}
			\item Modificación de todos los campos
			\item Actualización de imágenes
			\item Preservación del historial
		\end{itemize}
		
		\item \textbf{Eliminar Ejercicios}:
		\begin{itemize}
			\item Eliminación con confirmación
			\item Limpieza de referencias
		\end{itemize}
		
		\item \textbf{Búsqueda y Filtrado}:
		\begin{itemize}
			\item Búsqueda por nombre
			\item Filtrado por categoría
			\item Estadísticas por categoría
		\end{itemize}
	\end{enumerate}
	
	\subsection{Sistema de Imágenes}
	
	El sistema soporta dos métodos para manejar imágenes:
	
	\begin{enumerate}
		\item \textbf{URL Externa}: Los administradores pueden proporcionar URLs de imágenes externas
		\item \textbf{Subida de Archivos}: Sistema de subida a Supabase Storage con:
		\begin{itemize}
			\item Validación de tipo de archivo (solo imágenes)
			\item Límite de tamaño (5MB máximo)
			\item Generación automática de nombres únicos
			\item Vista previa en tiempo real
		\end{itemize}
	\end{enumerate}
	
	\subsection{Función de Subida de Imágenes}
	
	\begin{lstlisting}[language=JavaScript, caption=Función uploadExerciseImage]
		export async function uploadExerciseImage(formData: FormData) {
			const file = formData.get("file") as File
			
			if (!file) {
				return { error: "No se proporcionó ningún archivo" }
			}
			
			if (!file.type.startsWith("image/")) {
				return { error: "El archivo debe ser una imagen" }
			}
			
			if (file.size > 5 * 1024 * 1024) {
				return { error: "El archivo no debe superar los 5MB" }
			}
			
			try {
				const supabase = await createClient()
				const { data: { user }, error: userError } = await supabase.auth.getUser()
				
				if (userError || !user) {
					return { error: "Usuario no autenticado" }
				}
				
				const fileExt = file.name.split(".").pop()
				const fileName = `${Date.now()}-${Math.random().toString(36).substring(7)}.${fileExt}`
				const filePath = `exercise-images/${fileName}`
				
				const { data, error: uploadError } = await supabase.storage
				.from("exercise-images")
				.upload(filePath, file, {
					cacheControl: "3600",
					upsert: false,
				})
				
				if (uploadError) {
					return { error: "Error al subir la imagen" }
				}
				
				const { data: { publicUrl } } = supabase.storage
				.from("exercise-images")
				.getPublicUrl(filePath)
				
				return { success: true, imageUrl: publicUrl }
			} catch (error) {
				return { error: "Error de conexión con la base de datos" }
			}
		}
	\end{lstlisting}
	
	\section{Funciones de Base de Datos}
	
	\subsection{Función get\_all\_users\_with\_roles}
	
	Esta función permite a los administradores obtener información completa de todos los usuarios:
	
	\begin{lstlisting}[language=SQL, caption=Función get\_all\_users\_with\_roles]
		CREATE OR REPLACE FUNCTION get_all_users_with_roles()
		RETURNS TABLE (
		id uuid,
		email text,
		full_name text,
		role text,
		is_active boolean,
		is_professional boolean,
		created_at timestamptz
		)
		SECURITY DEFINER
		SET search_path = public
		LANGUAGE plpgsql
		AS $$
		BEGIN
		-- Check if the current user is an admin
		IF NOT EXISTS (
		SELECT 1 FROM user_roles ur_check
		WHERE ur_check.user_id = auth.uid() 
		AND ur_check.role = 'admin'
		) THEN
		RAISE EXCEPTION 'No autorizado';
		END IF;
		
		-- Return all users with their roles
		RETURN QUERY
		SELECT 
		au.id,
		au.email::text,
		COALESCE(
		au.raw_user_meta_data->>'full_name',
		CONCAT(
		COALESCE(au.raw_user_meta_data->>'first_name', ''),
		' ',
		COALESCE(au.raw_user_meta_data->>'last_name', '')
		),
		'Sin nombre'
		)::text,
		COALESCE(ur.role, 'user')::text,
		COALESCE(ur.is_active, true),
		COALESCE(ur.is_professional, false),
		au.created_at
		FROM auth.users au
		LEFT JOIN user_roles ur ON ur.user_id = au.id
		ORDER BY au.created_at DESC;
		END;
		$$;
	\end{lstlisting}
	
	\subsection{Función is\_admin}
	
	Función auxiliar para verificar privilegios administrativos:
	
	\begin{lstlisting}[language=SQL, caption=Función is\_admin]
		CREATE OR REPLACE FUNCTION public.is_admin(check_user_id UUID)
		RETURNS BOOLEAN AS $$
		DECLARE
		user_role TEXT;
		BEGIN
		SELECT role INTO user_role
		FROM public.user_roles
		WHERE user_id = check_user_id;
		
		RETURN user_role = 'admin';
		END;
		$$ LANGUAGE plpgsql SECURITY DEFINER;
	\end{lstlisting}
	
	\section{Seguridad y Row Level Security (RLS)}
	
	\subsection{Políticas de Seguridad}
	
	El sistema implementa múltiples capas de seguridad:
	
	\begin{enumerate}
		\item \textbf{RLS en user\_roles}:
		\begin{lstlisting}[language=SQL, caption=Políticas RLS para user\_roles]
			-- Admins can view all roles
			CREATE POLICY "Admins can view all roles" ON public.user_roles
			FOR SELECT USING (public.is_admin(auth.uid()));
			
			-- Admins can update roles
			CREATE POLICY "Admins can update roles" ON public.user_roles
			FOR UPDATE USING (public.is_admin(auth.uid()));
			
			-- Admins can insert roles
			CREATE POLICY "Admins can insert roles" ON public.user_roles
			FOR INSERT WITH CHECK (public.is_admin(auth.uid()));
		\end{lstlisting}
		
		\item \textbf{RLS en gym\_exercises}:
		\begin{lstlisting}[language=SQL, caption=Políticas RLS para gym\_exercises]
			-- Anyone can view gym exercises
			CREATE POLICY "Anyone can view gym exercises"
			ON gym_exercises
			FOR SELECT
			TO authenticated
			USING (true);
			
			-- Only admins can manage gym exercises
			CREATE POLICY "Only admins can manage gym exercises"
			ON gym_exercises
			FOR ALL
			TO authenticated
			USING (
			EXISTS (
			SELECT 1 FROM user_roles
			WHERE user_roles.user_id = auth.uid()
			AND user_roles.role = 'admin'
			)
			);
		\end{lstlisting}
	\end{enumerate}
	
	\subsection{Funciones SECURITY DEFINER}
	
	Las funciones marcadas como \texttt{SECURITY DEFINER} ejecutan con los privilegios del propietario de la función, permitiendo operaciones que de otro modo estarían restringidas por RLS.
	
	\section{Scripts de Configuración}
	
	\subsection{Creación del Usuario Administrador}
	
	Para establecer un usuario administrador, se debe ejecutar el siguiente script:
	
	\begin{lstlisting}[language=SQL, caption=Script para crear usuario administrador]
		-- Insert admin role for the admin user
		-- First, create user through Supabase Auth with email: juan@ejemplo.com and password: 123456
		-- Then run this script to assign the admin role
		
		INSERT INTO public.user_roles (user_id, role, is_active, approved_at)
		SELECT id, 'admin', true, NOW()
		FROM auth.users
		WHERE email = 'juan@ejemplo.com'
		ON CONFLICT (user_id) DO UPDATE
		SET role = 'admin', is_active = true, approved_at = NOW();
		
		-- Verify the admin was created
		SELECT u.id, u.email, ur.role, ur.is_active
		FROM auth.users u
		LEFT JOIN public.user_roles ur ON u.id = ur.user_id
		WHERE u.email = 'juan@ejemplo.com';
	\end{lstlisting}
	
	\subsection{Inserción de Ejercicios por Defecto}
	
	El sistema incluye ejercicios predefinidos para todas las categorías:
	
	\begin{lstlisting}[language=SQL, caption=Ejercicios por defecto]
		INSERT INTO gym_exercises (name, category, description) VALUES
		-- Pecho
		('Press de Banca', 'Pecho', 'Ejercicio básico para pecho con barra'),
		('Press Inclinado', 'Pecho', 'Press de banca en banco inclinado'),
		('Aperturas con Mancuernas', 'Pecho', 'Aperturas para pecho'),
		('Fondos en Paralelas', 'Pecho', 'Fondos para pecho y tríceps'),
		
		-- B\'iceps
		('Curl con Barra', 'B\'iceps', 'Curl de bíceps con barra recta'),
		('Curl con Mancuernas', 'B\'iceps', 'Curl alternado con mancuernas'),
		('Curl Martillo', 'B\'iceps', 'Curl con agarre neutro'),
		('Curl en Banco Scott', 'B\'iceps', 'Curl concentrado en banco'),
		
		-- Tr\'iceps
		('Press Francés', 'Tr\'iceps', 'Extensión de tríceps acostado'),
		('Fondos para Tr\'iceps', 'Tr\'iceps', 'Fondos en banco'),
		('Extensión en Polea', 'Tr\'iceps', 'Extensión de tríceps en polea alta'),
		('Patada de Tr\'iceps', 'Tr\'iceps', 'Extensión con mancuerna'),
		
		-- Hombros
		('Press Militar', 'Hombros', 'Press de hombros con barra'),
		('Elevaciones Laterales', 'Hombros', 'Elevaciones laterales con mancuernas'),
		('Elevaciones Frontales', 'Hombros', 'Elevaciones frontales'),
		('Pájaros', 'Hombros', 'Elevaciones posteriores'),
		
		-- Pierna
		('Sentadilla', 'Pierna', 'Sentadilla con barra'),
		('Prensa de Pierna', 'Pierna', 'Press de piernas en máquina'),
		('Peso Muerto', 'Pierna', 'Peso muerto convencional'),
		('Zancadas', 'Pierna', 'Zancadas con mancuernas'),
		('Extensión de Cuádriceps', 'Pierna', 'Extensión en máquina'),
		('Curl Femoral', 'Pierna', 'Curl de piernas acostado'),
		('Elevación de Gemelos', 'Pierna', 'Elevación de pantorrillas'),
		
		-- Espalda
		('Dominadas', 'Espalda', 'Dominadas con peso corporal'),
		('Remo con Barra', 'Espalda', 'Remo inclinado con barra'),
		('Remo con Mancuerna', 'Espalda', 'Remo a una mano'),
		('Jalón al Pecho', 'Espalda', 'Jalón en polea alta'),
		('Peso Muerto Rumano', 'Espalda', 'Peso muerto para espalda baja'),
		
		-- Otros
		('Plancha', 'Otros', 'Plancha abdominal'),
		('Abdominales', 'Otros', 'Crunch abdominal'),
		('Cardio', 'Otros', 'Ejercicio cardiovascular');
	\end{lstlisting}
	
	\section{Integración con Otros Módulos}
	
	\subsection{Sistema de Mensajería}
	
	El módulo de administrador se integra con el sistema de mensajería permitiendo:
	
	\begin{itemize}
		\item Gestionar qué usuarios son profesionales
		\item Activar/desactivar profesionales para el sistema de chat
		\item Controlar la visibilidad de usuarios en el sistema de mensajería
	\end{itemize}
	
	\subsection{Sistema de Ejercicios del Gimnasio}
	
	Los ejercicios gestionados por el administrador están disponibles para:
	
	\begin{itemize}
		\item Selección en rutinas de gimnasio
		\item Creación de historial de ejercicios
		\item Seguimiento de progreso de usuarios
	\end{itemize}
	
	\section{Consideraciones de Rendimiento}
	
	\subsection{Índices de Base de Datos}
	
	El sistema incluye índices optimizados:
	
	\begin{lstlisting}[language=SQL, caption=Índices para optimización]
		-- Índice para consultas de categoría de ejercicios
		CREATE INDEX idx_gym_exercises_category ON gym_exercises(category);
		
		-- Índice para consultas de usuarios profesionales
		CREATE INDEX idx_user_roles_is_professional 
		ON user_roles(is_professional) 
		WHERE is_professional = true;
	\end{lstlisting}
	
	\subsection{Optimizaciones de Frontend}
	
	\begin{itemize}
		\item \textbf{Lazy Loading}: Componentes se cargan bajo demanda
		\item \textbf{Debouncing}: Búsquedas con retraso para reducir consultas
		\item \textbf{Estados Optimizados}: Actualizaciones locales del estado
		\item \textbf{Revalidación}: Uso de \texttt{revalidatePath} para cache
	\end{itemize}
	
	\section{Testing y Debugging}
	
	\subsection{Logs de Debugging}
	
	El sistema incluye logging extensivo para debugging:
	
	\begin{lstlisting}[language=JavaScript, caption=Ejemplo de logging]
		console.log("[v0] Form data before submission:", formData)
		console.log("[v0] Image URL value:", formData.image_url)
		console.log("[v0] Server action result:", result)
	\end{lstlisting}
	
	\subsection{Manejo de Errores}
	
	\begin{itemize}
		\item \textbf{Try-Catch}: Captura de excepciones en todas las operaciones
		\item \textbf{Validación de Datos}: Verificación de entrada en cliente y servidor
		\item \textbf{Feedback Visual}: Notificaciones toast para todos los estados
		\item \textbf{Fallbacks}: Comportamiento degradado en caso de errores
	\end{itemize}
	
	\section{Configuración de Desarrollo}
	
	\subsection{Variables de Entorno}
	
	El módulo requiere las siguientes variables de entorno:
	
	\begin{lstlisting}[language=bash, caption=Variables de entorno necesarias]
		NEXT_PUBLIC_SUPABASE_URL=your_supabase_url
		NEXT_PUBLIC_SUPABASE_ANON_KEY=your_supabase_anon_key
		SUPABASE_SERVICE_ROLE_KEY=your_service_role_key
	\end{lstlisting}
	
	\subsection{Dependencias}
	
	Las principales dependencias del módulo incluyen:
	
	\begin{lstlisting}[language=json, caption=Dependencias principales]
		{
			"@supabase/supabase-js": "^2.38.0",
			"@radix-ui/react-select": "^2.0.0",
			"@radix-ui/react-switch": "^1.0.3",
			"@radix-ui/react-tabs": "^1.0.4",
			"lucide-react": "^0.292.0",
			"next": "14.0.0",
			"react": "^18.0.0",
			"tailwindcss": "^3.3.0"
		}
	\end{lstlisting}
	
	\section{Despliegue y Producción}
	
	\subsection{Consideraciones de Seguridad}
	
	\begin{enumerate}
		\item \textbf{RLS Habilitado}: Todas las tablas tienen RLS activado
		\item \textbf{Funciones SECURITY DEFINER}: Solo para operaciones específicas
		\item \textbf{Validación de Entrada}: Sanitización en cliente y servidor
		\item \textbf{Autenticación Requerida}: Todas las rutas protegidas
	\end{enumerate}
	
	\subsection{Monitoreo}
	
	\begin{itemize}
		\item \textbf{Logs de Supabase}: Monitoreo de consultas y errores
		\item \textbf{Analytics}: Seguimiento de uso del dashboard
		\item \textbf{Alertas}: Notificaciones de errores críticos
	\end{itemize}
	
	\section{Conclusión}
	
	El módulo de administrador de FitTrack representa una implementación robusta y segura de un sistema de gestión administrativa. Con su arquitectura bien estructurada, sistema de seguridad multicapa y interfaz de usuario moderna, proporciona las herramientas necesarias para gestionar eficientemente una aplicación de fitness.
	
	Las características clave incluyen:
	
	\begin{itemize}
		\item \textbf{Seguridad}: Implementación completa de RLS y funciones de seguridad
		\item \textbf{Escalabilidad}: Diseño modular que permite fácil extensión
		\item \textbf{Usabilidad}: Interfaz intuitiva con feedback visual inmediato
		\item \textbf{Mantenibilidad}: Código bien documentado y estructurado
		\item \textbf{Rendimiento}: Optimizaciones en base de datos y frontend
	\end{itemize}
	
	Este módulo sirve como base sólida para futuras expansiones y mejoras del sistema FitTrack.
	
\end{document}
