\documentclass[12pt,a4paper]{report}
\usepackage[utf8]{inputenc}
\usepackage[spanish]{babel}
\usepackage{listings}
\usepackage{color}
\usepackage{hyperref}
\usepackage{graphicx}
\usepackage{minted}
\usepackage{tcolorbox}
\usepackage{fancyhdr}

% Configuración de colores para código
\definecolor{codegreen}{rgb}{0,0.6,0}
\definecolor{codegray}{rgb}{0.5,0.5,0.5}
\definecolor{codepurple}{rgb}{0.58,0,0.82}
\definecolor{backcolour}{rgb}{0.95,0.95,0.92}

% Configuración de estilo de código
\lstdefinestyle{mystyle}{
    backgroundcolor=\color{backcolour},   
    commentstyle=\color{codegreen},
    keywordstyle=\color{magenta},
    numberstyle=\tiny\color{codegray},
    stringstyle=\color{codepurple},
    basicstyle=\ttfamily\footnotesize,
    breakatwhitespace=false,         
    breaklines=true,                 
    captionpos=b,                    
    keepspaces=true,                 
    numbers=left,                    
    numbersep=5pt,                  
    showspaces=false,                
    showstringspaces=false,
    showtabs=false,                  
    tabsize=2
}

\lstset{style=mystyle}

\title{Manual Técnico Detallado - FitTrack}
\author{Equipo de Desarrollo FitTrack}
\date{\today}

\begin{document}

\maketitle
\tableofcontents

\chapter{Módulos del Sistema}

\section{Módulo de Autenticación}
\subsection{Componentes Principales}
\begin{minted}{typescript}
// Ejemplo de login-form.tsx
export default function LoginForm() {
    const [email, setEmail] = useState('')
    const [password, setPassword] = useState('')
    
    const handleSubmit = async (e: React.FormEvent) => {
        e.preventDefault()
        const { data, error } = await signIn(email, password)
        // ...manejo de respuesta
    }
    // ...resto del componente
}
\end{minted}

\subsection{Integración con Supabase}
\begin{minted}{typescript}
// Ejemplo de auth-actions.ts
export async function signIn(email: string, password: string) {
    const supabase = createClient()
    return await supabase.auth.signInWithPassword({
        email,
        password
    })
}
\end{minted}

\section{Sistema de Base de Datos}
\subsection{Esquemas SQL}
\begin{minted}{sql}
-- Ejemplo de schema.sql
CREATE SCHEMA IF NOT EXISTS fittrack;

CREATE TABLE IF NOT EXISTS user_profiles (
    id UUID PRIMARY KEY REFERENCES auth.users(id),
    full_name TEXT,
    avatar_url TEXT,
    created_at TIMESTAMPTZ DEFAULT NOW(),
    updated_at TIMESTAMPTZ DEFAULT NOW()
);

CREATE TABLE IF NOT EXISTS workouts (
    id UUID DEFAULT uuid_generate_v4() PRIMARY KEY,
    user_id UUID REFERENCES auth.users(id),
    exercise_name TEXT NOT NULL,
    weight_kg NUMERIC,
    sets INTEGER,
    reps INTEGER,
    created_at TIMESTAMPTZ DEFAULT NOW()
);
\end{minted}

\subsection{Funciones y Triggers}
\begin{minted}{sql}
-- Ejemplo de functions.sql
CREATE OR REPLACE FUNCTION update_updated_at()
RETURNS TRIGGER AS $$
BEGIN
    NEW.updated_at = NOW();
    RETURN NEW;
END;
$$ LANGUAGE plpgsql;

CREATE TRIGGER set_timestamp
    BEFORE UPDATE ON user_profiles
    FOR EACH ROW
    EXECUTE FUNCTION update_updated_at();
\end{minted}

\chapter{Arquitectura del Sistema}

\section{Estructura de Archivos}
\begin{verbatim}
FitTrack/
+-- app/
|   +-- api/
|   +-- auth/
|   \-- ...
+-- components/
|   +-- ui/
|   +-- gym/
|   \-- ...
+-- lib/
|   \-- supabase/
\-- scripts/
\end{verbatim}

\section{Flujo de Datos}
\begin{figure}[h]
\centering
\includegraphics[width=0.8\textwidth]{data-flow-diagram.png}
\caption{Diagrama de Flujo de Datos}
\end{figure}

\chapter{API y Endpoints}

\section{Estructura de API Routes}
\begin{minted}{typescript}
// Ejemplo de API Route
export async function GET(request: Request) {
    const supabase = createRouteHandlerClient({ cookies })
    const { searchParams } = new URL(request.url)
    const userId = searchParams.get('userId')
    
    const { data, error } = await supabase
        .from('workouts')
        .select('*')
        .eq('user_id', userId)
        
    if (error) {
        return new Response(JSON.stringify({ error }), {
            status: 500
        })
    }
    
    return new Response(JSON.stringify(data))
}
\end{minted}

\chapter{Componentes UI}

\section{Sistema de Temas}
\begin{minted}{typescript}
// Ejemplo de ThemeProvider
export function ThemeProvider({ 
    children, 
    ...props 
}: ThemeProviderProps) {
    return (
        <NextThemesProvider {...props}>
            {children}
        </NextThemesProvider>
    )
}
\end{minted}

\section{Componentes Reutilizables}
\begin{minted}{typescript}
// Ejemplo de Button Component
export const Button = React.forwardRef<
    HTMLButtonElement,
    ButtonProps
>(({ 
    className, 
    variant = "default", 
    size = "default", 
    ...props 
}, ref) => {
    return (
        <button
            className={cn(
                buttonVariants({ variant, size, className })
            )}
            ref={ref}
            {...props}
        />
    )
})
\end{minted}

\chapter{Testing y Calidad}

\section{Pruebas Unitarias}
\begin{minted}{typescript}
// Ejemplo de test
describe('Button Component', () => {
    it('renders correctly', () => {
        render(<Button>Test</Button>)
        expect(screen.getByRole('button')).toBeInTheDocument()
    })
})
\end{minted}

\section{E2E Testing}
\begin{minted}{typescript}
// Ejemplo de E2E test
describe('Login Flow', () => {
    it('allows user to login', async () => {
        await page.goto('/login')
        await page.fill('input[type="email"]', 'test@example.com')
        await page.fill('input[type="password"]', 'password')
        await page.click('button[type="submit"]')
        await expect(page).toHaveURL('/dashboard')
    })
})
\end{minted}

\chapter{Seguridad}

\section{Políticas RLS}
\begin{minted}{sql}
-- Ejemplo de RLS
ALTER TABLE workouts ENABLE ROW LEVEL SECURITY;

CREATE POLICY "Users can view own workouts"
    ON workouts
    FOR SELECT
    USING (auth.uid() = user_id);
\end{minted}

\chapter{Mantenimiento y Despliegue}

\section{Scripts de Mantenimiento}
\begin{minted}{bash}
#!/bin/bash
# Ejemplo de script de backup
pg_dump -h localhost -U postgres fittrack > backup.sql
\end{minted}

\section{Proceso de Despliegue}
\begin{minted}{bash}
# Ejemplo de despliegue
npm run build
vercel deploy --prod
\end{minted}

\chapter{Apéndices}

\section{Glosario Técnico}
\begin{description}
    \item[RLS] Row Level Security
    \item[JWT] JSON Web Token
    \item[SSR] Server Side Rendering
\end{description}

\section{Referencias}
\begin{itemize}
    \item Next.js Documentation
    \item Supabase Documentation
    \item TypeScript Handbook
\end{itemize}

\chapter{Página de Gimnasio}

\section{Componentes de la Página de Gimnasio}
\begin{minted}{typescript}
import GymMetrics from '@/components/gym/gym-metrics'
import WorkoutForm from '@/components/gym/workout-form'
import WorkoutList from '@/components/gym/workout-list'
import ExerciseProgressTracker from '@/components/gym/exercise-progress-tracker'

export default function GymPage() {
  return (
    <div className="container mx-auto p-4 space-y-6">
      <h1 className="text-2xl font-bold">Panel de Ejercicios</h1>
      <GymMetrics />
      <div className="grid md:grid-cols-2 gap-6">
        <WorkoutForm />
        <ExerciseProgressTracker />
      </div>
      <WorkoutList />
    </div>
  )
}
\end{minted}

\end{document}